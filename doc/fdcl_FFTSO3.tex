\documentclass{ieeetran}
%\documentclass{ieeetran}
\usepackage{amssymb,amsmath,amsthm,graphicx,subfigure}
\usepackage{algpseudocode,cases,booktabs}
\usepackage{tikz}
\usetikzlibrary{arrows.meta}

\newcommand{\norm}[1]{\ensuremath{\left\| #1 \right\|}}
\newcommand{\bracket}[1]{\ensuremath{\left[ #1 \right]}}
\newcommand{\braces}[1]{\ensuremath{\left\{ #1 \right\}}}
\newcommand{\parenth}[1]{\ensuremath{\left( #1 \right)}}
\newcommand{\pair}[1]{\ensuremath{\langle #1 \rangle}}
\newcommand{\met}[1]{\ensuremath{\langle\langle #1 \rangle\rangle}}
\newcommand{\refeqn}[1]{(\ref{eqn:#1})}
\newcommand{\reffig}[1]{Figure \ref{fig:#1}}
\newcommand{\tr}[1]{\mathrm{tr}\ensuremath{\negthickspace\bracket{#1}}}
\newcommand{\trs}[1]{\mathrm{tr}\ensuremath{[#1]}}
\newcommand{\ave}[1]{\mathrm{E}\ensuremath{[#1]}}
\newcommand{\deriv}[2]{\ensuremath{\frac{\partial #1}{\partial #2}}}
\newcommand{\dderiv}[2]{\ensuremath{\dfrac{\partial #1}{\partial #2}}}
\newcommand{\SO}{\ensuremath{\mathsf{SO(3)}}}
\newcommand{\T}{\ensuremath{\mathsf{T}}}
\renewcommand{\L}{\ensuremath{\mathsf{L}}}
\newcommand{\so}{\ensuremath{\mathfrak{so}(3)}}
\newcommand{\SE}{\ensuremath{\mathsf{SE(3)}}}
\newcommand{\se}{\ensuremath{\mathfrak{se}(3)}}
\renewcommand{\Re}{\ensuremath{\mathbb{R}}}
\newcommand{\Cp}{\ensuremath{\mathbb{C}}}
\newcommand{\aSE}[2]{\ensuremath{\begin{bmatrix}#1&#2\\0&1\end{bmatrix}}}
\newcommand{\ase}[2]{\ensuremath{\begin{bmatrix}#1&#2\\0&0\end{bmatrix}}}
\newcommand{\D}{\ensuremath{\mathbf{D}}}
\renewcommand{\d}{\ensuremath{\mathfrak{d}}}
\newcommand{\Sph}{\ensuremath{\mathsf{S}}}
\renewcommand{\S}{\Sph}
\newcommand{\J}{\ensuremath{\mathbf{J}}}
\newcommand{\Ad}{\ensuremath{\mathrm{Ad}}}
\newcommand{\intp}{\ensuremath{\mathbf{i}}}
\newcommand{\extd}{\ensuremath{\mathbf{d}}}
\newcommand{\hor}{\ensuremath{\mathrm{hor}}}
\newcommand{\ver}{\ensuremath{\mathrm{ver}}}
\newcommand{\dyn}{\ensuremath{\mathrm{dyn}}}
\newcommand{\geo}{\ensuremath{\mathrm{geo}}}
\newcommand{\Q}{\ensuremath{\mathsf{Q}}}
\newcommand{\G}{\ensuremath{\mathsf{G}}}
\newcommand{\g}{\ensuremath{\mathfrak{g}}}
\newcommand{\Hess}{\ensuremath{\mathrm{Hess}}}

\newcommand{\bfi}{\bfseries\itshape\selectfont}
\DeclareMathOperator*{\argmax}{arg\,max}
\DeclareMathOperator*{\argmin}{arg\,min}

\date{}

\newtheorem{definition}{Definition}[section]
\newtheorem{lem}{Lemma}[section]
\newtheorem{prop}{Proposition}[section]
\newtheorem{remark}{Remark}[section]
\newtheorem{theorem}{Theorem}[section]

%\graphicspath{{./Figs/}}

\title{Noncommutative Harmonic Analysis on \SO}
\author{Taeyoung Lee%
\thanks{Taeyoung Lee, Mechanical and Aerospace Engineering, George Washington University, Washington DC 20052 {\tt tylee@gwu.edu}}
}




\begin{document}

\maketitle

\section{Harmonic Analysis on $\SO$}

\subsection{Representation on a Lie group}

Let $f\in\mathcal{L}^2(\Re^n)$. The (left) representation $U(g)$ of $\G$ is $\mathsf{GL}(\mathcal{L}^2(\Re^n))$, i.e., the set of linear transformation on $\mathcal{L}^2(\Re^n)$ defined such that
\begin{equation}\label{eqn:Ug}
(U(g)f)(x) = f(g^{-1} x).
\end{equation}
It is straightforward to show the representation is a homomorphism, 
\[
(U(g_1)(U(g_2)f))(x) = f(g_2^{-1} g_1^{-1} x) = (U(g_1g_2)f)(x).
\]
By selecting a basis for the invariance subspaces of $\mathcal{L}^2(\Re^n)$, $U(g)$ can be represented by a matrix, which is called a matrix representation of $\G$. 

Two matrix representations are \textit{equivalent}, if one can obtained by a similarity transform of the other. Any representation can be transformed into a \textit{unitary} representation by a similarity transform so that $U(g)U^*(g)=I$, or $U(g^{-1})=U^*(g)$.  A matrix representation is \textit{reducible}, if it is equivalent to the direct sum of others, or equivalently, it can be block-diagonalized. One can redefine \refeqn{Ug} with the group acting acting on the right side of $x$. Each irreducible representation acts only on the corresponding subspace, and the choice between the left action or the right action does not matter in the definition of the representation. 


\subsection{Irreducible Unitary Representation on $\SO$}

Since there is two-to-one homomorphism from $\mathsf{SU(2)}$ to $\SO$, the representation of $\SO$ should be a subset of those for $\mathsf{SU(2)}$, which is a set of linear transformation on $\mathcal{L}^2(\Cp^2)$. The set of homogeneous polynomials is a basis of $\mathcal{L}^2(\Cp^2)$, and one can find the matrix representation of $\mathsf{SU(2)}$ with \refeqn{Ug}, which results in generalizations of the associated Legendre function~\cite{ChiKya01} that is also shown to be unitary, and irreducible. An irreducible unitary representation of $\SO$ is constructed by taking the terms with integer indices. 

The wigner-D function is another irreducible unitary representation on $\mathsf{SU(2)}$ (or $\SO$) that is equivalent to the above representation based on the generalized associated Legendre function. Specifically, the wigner-D function is given by $D^l_{m,n}(R)\in\Cp$, where the index $l$ is a non-negative integer, and the integer indices $m,n$ vary in $-l\leq m,n \leq l$. When the low indices are dropped, $D^l(R)$ is considered as a square matrix where the row index (resp. the column index) corresponds to $m$ (resp. $n$) varying from $-l\leq m,n \leq l$. As such $D^l(R)\in\Cp^{2l+1,2l+1}$.

Let $(\alpha,\beta,\gamma)\in[0,2\pi]\times[0,\pi]\times[0,2\pi]$ be the 3-2-3 Euler angles, i.e., 
\[
R(\alpha,\beta,\gamma)=\exp(\alpha\hat e_3)\exp(\beta\hat e_2)\exp(\gamma\hat e_3). 
\]
Then, $D^l_{m,n}(R(\alpha,\beta,\gamma))$ is given by
\[
D^l_{m,n}(R(\alpha,\beta,\gamma)) = e^{-i m\alpha} d^l_{m,n}(\beta) e^{-i n\gamma},
\]
where $d^l_{m,n}(\beta)$ is a real-valued wigner-d function. A recursive formulation to evaluate the wigner-d function is presented in~\cite{BlaFloJMS97}. As $D^l(R)$ is unitary, we have $D^l(R)(D^{l}(R))^*=I_{2l+1}$ or equivalently $(D^l(R))^{-1}=(D^l(R))^*$. Furthermore, as it is a homomorphism, $D^l(R)D^l(R^T)=I_{2l+1}$. Therefore,
\begin{gather*}
D^l(R^T) = (D^l(R))^{-1} = (D^l(R))^*,\\
D^l(I_{3\times 3}) =I_{2l+1},\quad
D^l_{m,n}(R(0,0,0))=\delta_{m,n},
\end{gather*}
Since $d^l(\beta)=D^l(R(0,\beta,0))$, it follows
\begin{gather*}
d^l(-\beta) = (d^l(\beta))^{-1} = (d^l(\beta))^T,\\
d^l(0)=I_{2l+1},\quad d^l_{m,n}(0)=\delta_{m,n}.
\end{gather*}


While this formulation is based on the particular 3-2-3 Euler angles, 3-1-3 Euler angles can be used without any modification, as it will be equivalent anyway~\cite{ChiKya01}. Another nice feature is that $D^l_{m,n}$ is given by a product of three terms, which depend solely on each of three Euler angles. This follows directly from the use of a Gel`fand-Tsetlin basis, i.e., a basis that respects the decomposition of the restriction to the subgroup $\mathsf{SO(2)}$~\cite{MasRocGC97}. This is useful for devising fast Fourier transforms on $\SO$. Alternatively, the wigner-D matrix is formulated in terms of quaternions and rotation matrices in~\cite{LynStoMS89}.


We define an inner product on $\mathcal{L}^2(\SO)$ as
\[
\pair{f(R),g(R)}=\int_{\SO} f(R) g^*(R) dR.
\]
For the 3-2-3 Euler angles, the Haar measure is written as $dR = \frac{1}{8\pi^2}\sin\beta d\alpha d\beta d\gamma$ that is normalized such that $\int_{SO} dR = 1$. 

Due to the Peter-Weyl theorem, the collection of the irreducible unitary representations on $\SO$ form a complete orthogonal basis for $\mathcal{L}^2(\SO)$ over the above inner product. Specifically
\[
\pair{ D^{l_1}_{m_1,n_1}(R), D^{l_2}_{m_2,n_2}(R)} = \frac{1}{2l_1+1}\delta_{l_1,l_2}\delta_{m_1,m_2}\delta_{n_1,n_2}. 
\]
When $m_1=m_2$ and $n_1=n_2$, this reduces to
\[
\int_{0}^\pi d^{l_1}_{m,n}(\beta)d^{l_2}_{m,n}(\beta)\sin\beta d\beta = \frac{2}{2l_1+1}\delta_{l_1,l_2}. 
\]

The various identities, symmetry relatives, and derivatives of $D^{l}_{m,n}$ are provided in~\cite{VarMos88}. In particular,
\begin{align}
&\deriv{}{\beta} D^l_{m,n}(\alpha,\beta,\gamma)\nonumber\\
& = -\frac{1}{2}\sqrt{(l+m)(l-m+1)}e^{-i\alpha} D^l_{m-1,n}(\alpha,\beta,\gamma)\nonumber\\
& \quad + \frac{1}{2}\sqrt{(l-m)(l+m+1)}e^{i\alpha} D^l_{m+1,n}(\alpha,\beta,\gamma).\label{eqn:dD_dbeta}
\end{align}


\subsection{Character}

The \textit{character} of the representation is given by
\[
\chi^l(R) = \trs{D^l(R)}=\sum_{m=-l}^l e^{-im(\alpha+\gamma)} d^l_{m,m}(\beta).
\]
It is a \textit{class function} as it is invariant under the conjugation, i.e., for any $Q\in\SO$
\[
\chi(Q^T R Q) = \trs{D(Q^T R Q)}=\trs{D(Q^T)D(R)D(Q)}=\chi(R).
\]
As such, the character only depends on the rotation angle, i.e., when $R=\exp(\theta\hat r)$ for any $r\in\Sph^2$, 
\begin{align*}
&\chi^l(\exp(\theta\hat r))\\
&= \chi^l{(R(0,\theta,0))}= \sum_{m=-l}^l d^l_{m,m}(\theta)\\
&= \chi^l{(R(\theta,0,0))}= \sum_{m=-l}^l e^{-im\theta}
=\frac{\sin(\frac{2l+1}{2}\theta)}{\sin\frac{\theta}{2}}.
\end{align*}

Let a rotation matrix parameterized by spherical coordinates $(\lambda,\nu,\theta)\in[0,2\pi]\times[0,\pi]\times[0,\pi]$, where $\lambda$ and $\nu$ are the polar and azimuthal angles for the axis of rotation, and $\theta$ is the angle of rotation. In these coordinate, the normalized Haar measure is given by $dR=\frac{1}{2\pi^2} \sin^2\frac{\theta}{2} \sin\nu d\lambda d\nu d\theta$. We can show the orthogonality of the character as follows
\begin{align*}
&\pair{\chi^{l_1}(R), \chi^{l_2}(R)}
\\
&=\frac{1}{2\pi^2} \int_{0}^{2\pi} \int_{0}^\pi \int_0^\pi \sin(\frac{2l_1+1}{2}\theta)\sin(\frac{2l_2+1}{2}\theta) \sin\nu   d\theta  d\nu d\lambda\\
&=\frac{2}{\pi} \int_0^\pi \sin(\frac{2l_1+1}{2}\theta)\sin(\frac{2l_2+1}{2}\theta)  d\theta =\delta_{l_1,l_2}.
\end{align*}

It has been shown that a representation $\chi(R)$ is irreducible if and only if $\|\chi(R)\|=1$. Consequently, the above orthogonality implies that $D^l(R)$ is irreducible for any $l$.


\subsection{Operational Properties}

Given the matrix representation $D^l(R)$, we can define a representation of $\so\simeq \Re^3$ as
\[
u^l(\eta)= \frac{d}{d\epsilon}\bigg|_{\epsilon=0} D^l(\exp(\epsilon\hat\eta)),
\]
where $\eta\in\Re^3$. It satisfies
\[
u^l([\eta,\zeta])=[u^l(\eta),u^l(\zeta)].
\]
As it is a linear operator,
\[
u^l(\sum_{i=1}^3 \eta_i e_i) = \sum_{i=1}^3 \eta_i u^l(e_i).
\]
As such, we need to compute $u^l(e_1),u^l(e_2),u^l(e_3)\in\Cp^{(2l+1)\times(2l+1)}$ only.

Since $\exp(\epsilon\hat e_3) = R(\epsilon,0,0)$,
\begin{align}
u^l_{m,n}(e_3) & = \frac{d}{d\epsilon}\bigg|_{\epsilon=0} e^{-im\epsilon}d^l_{m,n}(0)=-im\delta_{m,n}.
\end{align}

Since $\exp(\epsilon\hat e_2) = R(0,\epsilon,0)$, 
\begin{align}
u^l_{m,n}(e_2) & = \frac{d}{d\epsilon}\bigg|_{\epsilon=0} D^l_{m,n}(R(0,\epsilon,0))\nonumber\\
& = -\frac{1}{2}\sqrt{(l+m)(l-m+1)} \delta_{m-1,n}\nonumber\\
& \quad + \frac{1}{2}\sqrt{(l-m)(l+m+1)} \delta_{m+1,n}\nonumber\\
& = -\frac{1}{2}c^l_n \delta_{m-1,n}+\frac{1}{2}c^l_{-n} \delta_{m+1,n}.
\end{align}
with $c^l_n=\sqrt{(l-n)(l+n+1)}$, which is obtained from \refeqn{dD_dbeta}.

Last, $u^l(e_1)$ can be obtained by $u^l(e_1)=u^l(e_2\times e_3)=[u^l(e_2), u^l(e_3)]$ as
\begin{align}
u^l_{m,n}(e_1) 
& = -\frac{1}{2}i c^l_n \delta_{m-1,n}-\frac{1}{2}i c^l_{-n} \delta_{m+1,n}.
\end{align}

\subsection{Fourier Transform on $\SO$}


Consequently, any $f\in\mathcal{L}^2(\SO)$ has the following decomposition
\begin{align}
f(R(\alpha,\beta,\gamma)) &= \sum_{l=0}^\infty \sum_{m,n=-l}^l (2l+1)\bar f^l_{m,n} D^l_{m,n}(\alpha,\beta,\gamma)\nonumber\\
&= \sum_{l=0}^\infty (2l+1)\trs{(\bar f^l)^T D^l(\alpha,\beta,\gamma)},\label{eqn:f_IFT}
\end{align}
where $\bar f^l_{m,n}\in\Cp$ is obtained by
\begin{align}
& \bar f^l_{m,n}=\pair{ f(R), D^l_{m,n}(R)}\nonumber\\
& =\frac{1}{8\pi^2}\int_0^{2\pi}\int_{0}^\pi\int_0^{2\pi} f(R(\alpha,\beta,\gamma)) (D^{l}_{m,n}(R(\alpha,\beta,\gamma)))^*d\alpha d\beta d\gamma.\label{eqn:f_FT}
\end{align}
Several variations of the above definition exists. For example, in~\cite{KosRocJFAA08}, $D^l_{m,n}$ is normalized such that the factor $2l+1$ does not appear.

We also have the Plancherel theorem:
\[
\pair{f_1(R), f_2(R)} = \sum_{l=0}^\infty (2l+1)\pair{\bar f_1^l, \bar f_2^l},
\]
with the inner product $\pair{A,B}=\trs{A^*B}$ on $A,B\in\Cp^{n\times n}$. Also, 
\[
\| f(R)\|^2 = \sum_{l=0}^\infty (2l+1)\|\bar f^l\|^2
\]



\subsection{Sampling}

A function $f\in\mathcal{L}^2(\SO)$ is called \textit{band-limited} with the band $l_{\max}$ if $f^l=0$ for any $l>l_{\max}$ in \refeqn{f_IFT}.

Consider the following uniform grid for $(\alpha,\beta,\gamma)\in[0,2\pi]\times[0,\pi]\times[0,2\pi]$.

\bibliography{/Users/tylee/Documents/BibMaster17}
\bibliographystyle{IEEEtran}

\end{document}
