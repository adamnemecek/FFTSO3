\documentclass[onecolumn,11pt]{IEEEtran}
%\documentclass{ieeetran}
\usepackage{amssymb,amsmath,amsthm,graphicx,subfigure}
\usepackage{algpseudocode,cases,booktabs}
\usepackage{tikz}
\usetikzlibrary{arrows.meta}
\usepackage{mathdots}
\usepackage{array,threeparttable,multirow}
\usepackage[math]{cellspace}

\newcommand{\norm}[1]{\ensuremath{\left\| #1 \right\|}}
\newcommand{\bracket}[1]{\ensuremath{\left[ #1 \right]}}
\newcommand{\braces}[1]{\ensuremath{\left\{ #1 \right\}}}
\newcommand{\parenth}[1]{\ensuremath{\left( #1 \right)}}
\newcommand{\pair}[1]{\ensuremath{\langle #1 \rangle}}
\newcommand{\met}[1]{\ensuremath{\langle\langle #1 \rangle\rangle}}
\newcommand{\refeqn}[1]{(\ref{eqn:#1})}
\newcommand{\reffig}[1]{Figure \ref{fig:#1}}
\newcommand{\tr}[1]{\mathrm{tr}\ensuremath{\negthickspace\bracket{#1}}}
\newcommand{\trs}[1]{\mathrm{tr}\ensuremath{[#1]}}
\newcommand{\ave}[1]{\mathrm{E}\ensuremath{[#1]}}
\newcommand{\deriv}[2]{\ensuremath{\frac{\partial #1}{\partial #2}}}
\newcommand{\dderiv}[2]{\ensuremath{\dfrac{\partial #1}{\partial #2}}}
\newcommand{\SO}{\ensuremath{\mathsf{SO(3)}}}
\newcommand{\T}{\ensuremath{\mathsf{T}}}
\renewcommand{\L}{\ensuremath{\mathsf{L}}}
\newcommand{\so}{\ensuremath{\mathfrak{so}(3)}}
\newcommand{\SE}{\ensuremath{\mathsf{SE(3)}}}
\newcommand{\se}{\ensuremath{\mathfrak{se}(3)}}
\renewcommand{\Re}{\ensuremath{\mathbb{R}}}
\newcommand{\Cp}{\ensuremath{\mathbb{C}}}
\newcommand{\aSE}[2]{\ensuremath{\begin{bmatrix}#1&#2\\0&1\end{bmatrix}}}
\newcommand{\ase}[2]{\ensuremath{\begin{bmatrix}#1&#2\\0&0\end{bmatrix}}}
\newcommand{\D}{\ensuremath{\mathbf{D}}}
\renewcommand{\d}{\ensuremath{\mathfrak{d}}}
\newcommand{\Sph}{\ensuremath{\mathsf{S}}}
\renewcommand{\S}{\Sph}
\newcommand{\J}{\ensuremath{\mathbf{J}}}
\newcommand{\Ad}{\ensuremath{\mathrm{Ad}}}
\newcommand{\intp}{\ensuremath{\mathbf{i}}}
\newcommand{\extd}{\ensuremath{\mathbf{d}}}
\newcommand{\hor}{\ensuremath{\mathrm{hor}}}
\newcommand{\ver}{\ensuremath{\mathrm{ver}}}
\newcommand{\dyn}{\ensuremath{\mathrm{dyn}}}
\newcommand{\geo}{\ensuremath{\mathrm{geo}}}
\newcommand{\Q}{\ensuremath{\mathsf{Q}}}
\newcommand{\G}{\ensuremath{\mathsf{G}}}
\newcommand{\g}{\ensuremath{\mathfrak{g}}}
\newcommand{\Hess}{\ensuremath{\mathrm{Hess}}}

\newcommand{\bfi}{\bfseries\itshape\selectfont}
\DeclareMathOperator*{\argmax}{arg\,max}
\DeclareMathOperator*{\argmin}{arg\,min}

\date{}

\newtheorem{definition}{Definition}[section]
\newtheorem{lem}{Lemma}[section]
\newtheorem{prop}{Proposition}[section]
\newtheorem{remark}{Remark}[section]
\newtheorem{theorem}{Theorem}[section]

%\graphicspath{{./Figs/}}

\title{Noncommutative Harmonic Analysis on \SO}
\author{Taeyoung Lee%
\thanks{Taeyoung Lee, Mechanical and Aerospace Engineering, George Washington University, Washington DC 20052 {\tt tylee@gwu.edu}}
}

\begin{document}

\maketitle

\section{Harmonic Analysis on $\SO$}

\subsection{Representation on a Lie group}

Let $f\in\mathcal{L}^2(\Re^n)$. The (left) representation $U(g)$ of $\G$ is $\mathsf{GL}(\mathcal{L}^2(\Re^n))$, i.e., the set of linear transformation on $\mathcal{L}^2(\Re^n)$ defined such that
\begin{equation}\label{eqn:Ug}
(U(g)f)(x) = f(g^{-1} x).
\end{equation}
It is straightforward to show the representation is a homomorphism, 
\[
(U(g_1)(U(g_2)f))(x) = f(g_2^{-1} g_1^{-1} x) = (U(g_1g_2)f)(x).
\]
By selecting a basis for the invariance subspaces of $\mathcal{L}^2(\Re^n)$, $U(g)$ can be represented by a matrix, which is called a matrix representation of $\G$. 

Two matrix representations are \textit{equivalent}, if one can obtained by a similarity transform of the other. 
Any representation can be transformed into a \textit{unitary} representation by a similarity transform so that $U(g)U^*(g)=I$, or $U(g^{-1})=U^*(g)$. 
A matrix representation is \textit{reducible}, if it is equivalent to the direct sum of others, or equivalently, it can be block-diagonalized. 
One can redefine \refeqn{Ug} with the group acting acting on the right side of $x$. 
Each irreducible representation acts only on the corresponding subspace, and the choice between the left action or the right action does not matter in the definition of the representation. 

\subsection{Irreducible Unitary Representation on $\SO$}

Since there is two-to-one homomorphism from $\mathsf{SU(2)}$ to $\SO$, the representation of $\SO$ should be a subset of those for $\mathsf{SU(2)}$, which is a set of linear transformation on $\mathcal{L}^2(\Cp^2)$. 
The set of homogeneous polynomials is a basis of $\mathcal{L}^2(\Cp^2)$, and one can find the matrix representation of $\mathsf{SU(2)}$ with \refeqn{Ug}, which results in generalizations of the associated Legendre function~\cite{ChiKya01} that is also shown to be unitary, and irreducible. 
An irreducible unitary representation of $\SO$ is constructed by taking the terms with integer indices. 

The wigner-D function is another irreducible unitary representation on $\mathsf{SU(2)}$ (or $\SO$) that is equivalent to the above representation based on the generalized associated Legendre function. 
Specifically, the wigner-D function is given by $D^l_{m,n}(R)\in\Cp$, where the index $l$ is a non-negative integer, and the integer indices $m,n$ vary in $-l\leq m,n \leq l$. 
When the low indices are dropped, $D^l(R)$ is considered as a square matrix where the row index (resp. the column index) corresponds to $m$ (resp. $n$) varying from $-l\leq m,n \leq l$. 
As such $D^l(R)\in\Cp^{2l+1,2l+1}$.

Let $(\alpha,\beta,\gamma)\in[0,2\pi)\times[0,\pi]\times[0,2\pi)$ be the 3-2-3 Euler angles, i.e., 
\[
R(\alpha,\beta,\gamma)=\exp(\alpha\hat e_3)\exp(\beta\hat e_2)\exp(\gamma\hat e_3). 
\]
Then, $D^l_{m,n}(R(\alpha,\beta,\gamma))$ is given by
\begin{equation}\label{eqn:Dlmn}
D^l_{m,n}(R(\alpha,\beta,\gamma)) = e^{-i m\alpha} d^l_{m,n}(\beta) e^{-i n\gamma},
\end{equation}
where $d^l_{m,n}(\beta)$ is a real-valued wigner-d function. 
A recursive formulation to evaluate the wigner-d function is presented in~\cite{BlaFloJMS97}. 

As $D^l(R)$ is unitary, we have $D^l(R)(D^{l}(R))^*=I_{2l+1}$ or equivalently $(D^l(R))^{-1}=(D^l(R))^*$. 
Furthermore, as it is a homomorphism, $D^l(R)D^l(R^T)=I_{2l+1}$. 
Therefore,
\begin{gather*}
D^l(R^T) = (D^l(R))^{-1} = (D^l(R))^*,\\
D^l(I_{3\times 3}) =I_{2l+1},\quad
D^l_{m,n}(R(0,0,0))=\delta_{m,n},
\end{gather*}
Since $d^l(\beta)=D^l(R(0,\beta,0))$, it follows
\begin{gather*}
d^l(-\beta) = (d^l(\beta))^{-1} = (d^l(\beta))^T,\\
d^l(0)=I_{2l+1},\quad d^l_{m,n}(0)=\delta_{m,n}.
\end{gather*}
Also,
\begin{equation}\label{eqn:Dlmn_bar}
\overline{D^l_{m,n}(R(\alpha,\beta,\gamma))}=(-1)^{m-n} D^l_{-m,-n}(R(\alpha,\beta,\gamma)).
\end{equation}

While this formulation is based on the particular 3-2-3 Euler angles, 3-1-3 Euler angles can be used without any modification, as it will be equivalent anyway~\cite{ChiKya01}. 
Another nice feature is that $D^l_{m,n}$ is given by a product of three terms, which depend solely on each of three Euler angles. 
This follows directly from the use of a Gel`fand-Tsetlin basis, i.e., a basis that respects the decomposition of the restriction to the subgroup $\mathsf{SO(2)}$~\cite{MasRocGC97}. 
This is useful for devising fast Fourier transforms on $\SO$. 
Alternatively, the wigner-D matrix is formulated in terms of quaternions and rotation matrices in~\cite{LynStoMS89}.

We define an inner product on $\mathcal{L}^2(\SO)$ as
\[
    \pair{f(R),g(R)}=\int_{\SO} \overline{f(R)} g(R) dR.
\]
For the 3-2-3 Euler angles, the Haar measure is written as $dR = \frac{1}{8\pi^2}\sin\beta d\alpha d\beta d\gamma$ that is normalized such that $\int_{SO} dR = 1$. 

According to the Peter-Weyl theorem, the collection of the irreducible unitary representations on $\SO$ form a complete orthogonal basis for $\mathcal{L}^2(\SO)$ over the above inner product. 
Specifically
\begin{equation}
\pair{ D^{l_1}_{m_1,n_1}(R), D^{l_2}_{m_2,n_2}(R)} = \frac{1}{2l_1+1}\delta_{l_1,l_2}\delta_{m_1,m_2}\delta_{n_1,n_2}. \label{eqn:D_ortho}
\end{equation}
When $m_1=m_2$ and $n_1=n_2$, this reduces to
\[
\int_{0}^\pi d^{l_1}_{m,n}(\beta)d^{l_2}_{m,n}(\beta)\sin\beta d\beta = \frac{2}{2l_1+1}\delta_{l_1,l_2}. 
\]

The various identities, symmetry relatives, and derivatives of $D^{l}_{m,n}$ are provided in~\cite{VarMos88}. 
In particular,
\begin{align}
&\deriv{}{\beta} D^l_{m,n}(\alpha,\beta,\gamma)\nonumber\\
& = -\frac{1}{2}\sqrt{(l+m)(l-m+1)}e^{-i\alpha} D^l_{m-1,n}(\alpha,\beta,\gamma)\nonumber\\
& \quad + \frac{1}{2}\sqrt{(l-m)(l+m+1)}e^{i\alpha} D^l_{m+1,n}(\alpha,\beta,\gamma).\label{eqn:dD_dbeta}
\end{align}


\subsection{Character}

The \textit{character} of the representation is given by
\[
\chi^l(R) = \trs{D^l(R)}=\sum_{m=-l}^l e^{-im(\alpha+\gamma)} d^l_{m,m}(\beta).
\]
It is a \textit{class function} as it is invariant under the conjugation, i.e., for any $Q\in\SO$
\[
\chi(Q^T R Q) = \trs{D(Q^T R Q)}=\trs{D(Q^T)D(R)D(Q)}=\chi(R).
\]
As such, the character only depends on the rotation angle, i.e., when $R=\exp(\theta\hat r)$ for any $r\in\Sph^2$, 
\begin{align*}
&\chi^l(\exp(\theta\hat r))\\
&= \chi^l{(R(0,\theta,0))}= \sum_{m=-l}^l d^l_{m,m}(\theta)\\
&= \chi^l{(R(\theta,0,0))}= \sum_{m=-l}^l e^{-im\theta}
=\frac{\sin(\frac{2l+1}{2}\theta)}{\sin\frac{\theta}{2}}.
\end{align*}

Let a rotation matrix parameterized by spherical coordinates $(\lambda,\nu,\theta)\in[0,2\pi]\times[0,\pi]\times[0,\pi]$, where $\lambda$ and $\nu$ are the polar and azimuthal angles for the axis of rotation, and $\theta$ is the angle of rotation. 
In these coordinate, the normalized Haar measure is given by $dR=\frac{1}{2\pi^2} \sin^2\frac{\theta}{2} \sin\nu d\lambda d\nu d\theta$. 
We can show the orthogonality of the character as follows
\begin{align*}
&\pair{\chi^{l_1}(R), \chi^{l_2}(R)}
\\
&=\frac{1}{2\pi^2} \int_{0}^{2\pi} \int_{0}^\pi \int_0^\pi \sin(\frac{2l_1+1}{2}\theta)\sin(\frac{2l_2+1}{2}\theta) \sin\nu   d\theta  d\nu d\lambda\\
&=\frac{2}{\pi} \int_0^\pi \sin(\frac{2l_1+1}{2}\theta)\sin(\frac{2l_2+1}{2}\theta)  d\theta =\delta_{l_1,l_2}.
\end{align*}

It has been shown that a representation $D(R)$ is irreducible if and only if $\|\chi(R)\|=1$. Consequently, the above orthogonality implies that $D^l(R)$ is irreducible for any $l$.


\subsection{Operational Properties}

Given the matrix representation $D^l(R)$, we can define a representation of $\so\simeq \Re^3$ as
\[
u^l(\eta)= \frac{d}{d\epsilon}\bigg|_{\epsilon=0} D^l(\exp(\epsilon\hat\eta)),
\]
where $\eta\in\Re^3$. It satisfies
\[
u^l([\eta,\zeta])=[u^l(\eta),u^l(\zeta)].
\]
As it is a linear operator,
\[
u^l(\sum_{i=1}^3 \eta_i e_i) = \sum_{i=1}^3 \eta_i u^l(e_i).
\]
As such, we need to compute $u^l(e_1),u^l(e_2),u^l(e_3)\in\Cp^{(2l+1)\times(2l+1)}$ only.

Since $\exp(\epsilon\hat e_3) = R(\epsilon,0,0)$,
\begin{align}
u^l_{m,n}(e_3) & = \frac{d}{d\epsilon}\bigg|_{\epsilon=0} e^{-im\epsilon}d^l_{m,n}(0)=-im\delta_{m,n}.
\end{align}

Since $\exp(\epsilon\hat e_2) = R(0,\epsilon,0)$, 
\begin{align}
u^l_{m,n}(e_2) & = \frac{d}{d\epsilon}\bigg|_{\epsilon=0} D^l_{m,n}(R(0,\epsilon,0))\nonumber\\
& = -\frac{1}{2}\sqrt{(l+m)(l-m+1)} \delta_{m-1,n}\nonumber\\
& \quad + \frac{1}{2}\sqrt{(l-m)(l+m+1)} \delta_{m+1,n}\nonumber\\
& = -\frac{1}{2}c^l_n \delta_{m-1,n}+\frac{1}{2}c^l_{-n} \delta_{m+1,n}.
\end{align}
with $c^l_n=\sqrt{(l-n)(l+n+1)}$, which is obtained from \refeqn{dD_dbeta}.

Last, $u^l(e_1)$ can be obtained by $u^l(e_1)=u^l(e_2\times e_3)=[u^l(e_2), u^l(e_3)]$ as
\begin{align}
u^l_{m,n}(e_1) 
& = -\frac{1}{2}i c^l_n \delta_{m-1,n}-\frac{1}{2}i c^l_{-n} \delta_{m+1,n}.
\end{align}

\subsection{Fourier Transform on $\SO$}

According to the Peter-Weyl theorem, any $f\in\mathcal{L}^2(\SO)$ has the following decomposition
\begin{align}
f(R(\alpha,\beta,\gamma)) &= \sum_{l=0}^\infty \sum_{m,n=-l}^l (2l+1)\tilde f^l_{m,n} D^l_{m,n}(\alpha,\beta,\gamma)\nonumber\\
&= \sum_{l=0}^\infty (2l+1)\trs{(\bar f^l)^T D^l(\alpha,\beta,\gamma)},\label{eqn:f_IFT}
\end{align}
which is the Fourier transform. 
From the orthogonality property \refeqn{D_ortho}, the Fourier parameter $\tilde f^l_{m,n}\in\Cp$ is obtained by
\begin{align}
& \tilde f^l_{m,n}=\pair{ f(R), D^l_{m,n}(R)}\nonumber\\
& =\frac{1}{8\pi^2}\int_0^{2\pi}\int_{0}^\pi\int_0^{2\pi} f(R(\alpha,\beta,\gamma)) \overline{D^{l}_{m,n}(R(\alpha,\beta,\gamma))}d\alpha d\beta d\gamma,\label{eqn:f_FT}
\end{align}
which is the inverse transform. 
Several variations of the above definition exist. 
For example, in~\cite{KosRocJFAA08}, $D^l_{m,n}$ is normalized such that the factor $2l+1$ does not appear. 
We follow the convention of~\cite{ChiKya01}, and the factor appears in \refeqn{f_IFT}.

We also have the Plancherel theorem, stating
\[
\pair{f_1(R), f_2(R)} = \sum_{l=0}^\infty (2l+1)\pair{\bar f_1^l, \bar f_2^l},
\]
with the inner product $\pair{A,B}=\trs{A^*B}$ on $A,B\in\Cp^{n\times n}$. Also, 
\[
\| f(R)\|^2 = \sum_{l=0}^\infty (2l+1)\|\bar f^l\|^2
\]


\subsection{Sampling}

A function $f\in\mathcal{L}^2(\SO)$ is called \textit{band-limited} with the band $B$ if $\tilde f^l=0$ for any $l\geq B$ in \refeqn{f_IFT}, or equivalently
\begin{align}
f(R(\alpha,\beta,\gamma)) &= \sum_{l=0}^{B-1} \sum_{m,n=-l}^l (2l+1)\tilde f^l_{m,n} D^l_{m,n}(\alpha,\beta,\gamma).\label{eqn:fB}
\end{align}
The classical sampling theorem stated that the Fourier transform of a band limited function can be recovered from the sample values of the function that are chosen at a uniform grid with with a frequency that is at least twice of the band limit. 
Consequently, the Fourier transform of the above function can be computed from finite samples. 

Consider the following uniform grid for $(\alpha,\beta,\gamma)\in[0,2\pi)\times[0,\pi]\times[0,2\pi)$:
\[
[\SO]_d=\{(\alpha_{j_1},\beta_k,\gamma_{j_2})\,|\, j_1,j_2,k\in\{0,\ldots, 2B-1\}\},
\]
with
\begin{equation}\label{eqn:grid}
\alpha_j=\gamma_j= \frac{\pi j}{B},\quad \beta_k = \frac{\pi (2k+1)}{4B}.
\end{equation}

Define a sampling distribution of bandwidth $B$ as
\begin{equation}
s(R(\alpha,\beta,\gamma))=\sum_{j_1,k,j_2=0}^{2B-1} w_k \delta_{R(\alpha,\beta,\gamma),R(\alpha_{j_1},\beta_k,\gamma_{j_2})},
\end{equation}
which is the linear combination of grid points weighted by the parameter $w_k$. 
The Fourier transform of $s_B$ is given by
\begin{align*}
\bar s^l_{m,n} & = \pair{s(R(\alpha,\beta,\gamma)), D^l_{m,n}(R)}\\
&=\sum_{j_1,k,j_2=0}^{2B-1} w_k \overline{D^l_{m,n}(R(\alpha_{j_1},\beta_k,\gamma_{j_2}))}\\
& = \sum_{j_1=0}^{2B-1} e^{im\alpha_{j_1}}
\sum_{j_2=0}^{2B-1} e^{in\gamma_{j_2}}
\sum_{k=0}^{2B-1} w_k d^l_{m,n}(\beta_k).
\end{align*}
For the selected grid, $\sum_{j_1=0}^{2B-1} e^{im\alpha_{j_1}}=2B \delta_{m,0}$. 
Therefore, it reduces to
\begin{align*}
\bar s^l_{m,n}  
& = 4B^2 \delta_{m,0}\delta_{n,0}\sum_{k=0}^{2B-1} w_k d^l_{0,0}(\beta_k),\\
& = 4B^2 \delta_{m,0}\delta_{n,0}\sum_{k=0}^{2B-1} w_k P_l(\beta_k),
\end{align*}
where $P_l$ is the $l$-th Legendre polynomial. 
We select the weight such that
\begin{equation}
\sum_{k=0}^{2B-1} w_k P_l(\beta_k) = \frac{1}{4B^2} \delta_{l,0},\quad l=0,\ldots 2B-1.
\end{equation}
In~\cite{DriHeaAAM94}, it is explicitly given as
\begin{equation}
w_k = \frac{1}{4B^3}\sin\beta_k \sum_{j=0}^{B-1} \frac{1}{2j+1}\sin((2j+1)\beta_k).
\end{equation}
Substituting this, the Fourier transform of the sampling distribution is given by
\begin{equation}
\bar s ^l_{m,n}=\delta_{m,0}\delta_{n,0}\delta_{l,0},\quad l=0,\ldots 2B-1.
\label{eqn:bar_s}
\end{equation}

Define $f_s(R)=f(R) s(R)$. 
It is straightforward to show
\begin{align*}
&f_s(R(\alpha,\beta,\gamma))\\
& = 
\sum_{j_1,k,j_2=0}^{2B-1} w_k f(R(\alpha_{j_1},\beta_k,\gamma_{j_2})) 
\delta_{R(\alpha,\beta,\gamma),R(\alpha_{j_1},\beta_k,\gamma_{j_2})},
\end{align*}
and the Fourier transform is given by
\begin{align*}
&(\bar{f_s})^l_{m,n}\\
& = 
\sum_{j_1,k,j_2=0}^{2B-1} w_k f(R(\alpha_{j_1},\beta_k,\gamma_{j_2})) 
\overline{D^l_{m,n}(R(\alpha_{j_1},\beta_k,\gamma_{j_2}))}.
\end{align*}
Next, we show that this corresponds to the Fourier coefficient of $f$ within the band limit. 

From \refeqn{bar_s}, the sampling distribution can be expanded as
\[
s(R)=1+\sum_{l=2B}^\infty \sum_{m,n=-l}^l \tilde s^l_{m,n} D^l_{m,n}(R).
\]
Then,
\begin{align*}
f_s(R)&=f(R)+f(R)\sum_{l=2B}^\infty \sum_{m,n=-l}^l \tilde s^l_{m,n} D^l_{m,n}(R).
\end{align*}
Since $f(R)$ can be expanded as a linear combination of $D^{l_1}$ for $0\leq l_1 \leq B-1$. 
The last term of the above equation is expanded by the product $D^{l_1}D^{l_2}$ with $0\leq l_1 \leq B-1$  and $2B\leq l_2$. 
According to the Clebsch-Gordon theorem, $D^{l_1}D^{l_2}$ is a linear combination of $D^{l_3}$ for $|l_1-l_2|\leq l_3 \leq l_1+l_2$. 
We have $\min|l_1-l_2|=2B-1-B=B+1$. 
As such, $f$ and $f_s$ share the Fourier coefficients in the given band limit. 
Or equivalently, 
\begin{align}
&\tilde f^l_{m,n}\nonumber\\
& = 
\sum_{j_1,k,j_2=0}^{2B-1} w_k f(R(\alpha_{j_1},\beta_k,\gamma_{j_2})) 
\overline{D^l_{m,n}(R(\alpha_{j_1},\beta_k,\gamma_{j_2}))}.
\end{align}

\subsection{Fast Fourier Transform}
The above summation can be decomposed into
\begin{align*}
f^l_{m,n}&=\sum_{j_1,j_2,k=0}^{2B-1} w_k f(\alpha_{j_1},\beta_k,\gamma_{j_2})
e^{im\alpha_{j_1}} d^l_{m,n}(\beta_k) e^{in\gamma_{j_2}}\\
&=\sum_{j_1=0}^{2B-1} 
e^{im\alpha_{j_1}}\sum_{j_2=0}^{2B-1} e^{in\gamma_{j_2}} \sum_{k}^{2B-1}w_k f(\alpha_{j_1},\beta_k,\gamma_{j_2})d^l_{m,n}(\beta_k) \\
\end{align*}
This can be computed in the following order:
\begin{align*}
F^{\beta}_{l,m,n}(j_1,j_2)&=\sum_{k=0}^{2B-1}w_k f(\alpha_{j_1},\beta_k,\gamma_{j_2})d^l_{m,n}(\beta_k),\\
F^{\gamma}_{l,m,n}(j_1)&=\sum_{j_2=0}^{2B-1}e^{\frac{i 2 n \pi j_2}{2B}}F^{\beta}_{l,m,n}(j_1,j_2),\\
f^l_{m,n} & = \sum_{j_1=0}^{2B-1} e^{\frac{i 2 m\pi j_1}{2B}}F^{\gamma}_{l,m,n}(j_1),
\end{align*}
where the last two operations are Fourier transforms in a linear space, which can be computed by various FFT algorithms. 

%\paragraph*{Cooley-Tukey FFT}
%
%Here, we consider the following variation of the Cooley-Tukey algorithm~\cite{CooTukMC65}, referred to as radix-2 DIT that is applied when $2B=2^{b}\triangleq N$ for an integer $b$.
%\[
%f^l_{m,n} = \sum_{j_1=0}^{N-1} e^{\frac{i 2 m\pi j_1}{N}}F^{\gamma}_{l,m,n}(j_1),
%\]
%The sum is divided into the even indices and the odd indices as
%\begin{align*}
%f^l_{m,n} & = \sum_{j_1=0}^{N/2-1} e^{\frac{i 2 m\pi j_1}{N/2}}F^{\gamma}_{l,m,n}(2j_1)\\
%&\quad +e^{\frac{i 2 m\pi }{N}}\sum_{j_1=0}^{N/2-1} e^{\frac{i 2 m\pi  j_1}{N/2}}F^{\gamma}_{l,m,n}(2j_1+1)
%\\
%& \triangleq E_{l,m,n} + e^{\frac{i 2 m\pi }{N}}  O_{l,m,n}.
%\end{align*}
%As such, 
%\begin{align*}
%f^l_{m,n} &= E_{l,m,n} + e^{\frac{i 2 m\pi }{N}}  O_{l,m,n}\\
%f^l_{m+N/2,n} &= E_{l,m,n} + e^{\frac{i 2 m\pi }{N}}  O_{l,m,n}\\
%X_{k+N/2} & = E_k - e^{\frac{2\pi i }{N}k}  O_k.
%\end{align*}
%These transforms the original DFT of the length $N$ to two DFTs of the length $N/2$. It can be recursively applied when $N$ is a power of $2$. 

\subsection{Clebsch-Gordon Coefficients}

The product of two wigner D matrices can be transformed, and the resulting Fourier coefficients are referred to as Clebsch-Gordon coefficients. 
It has been shown that
\begin{align}
    D^{l_1}_{m_1,n_1} (R) D^{l_2}_{m_2,n_2}(R) = \sum_{l=|l_1-l_2|}^{l_1+l_2} \sum_{m,n=-l}^l C^{l,m}_{l_1,m_1,l_2,m_2} C^{l,n}_{l_1,n_1,l_2,n_2} D^{l}_{m,n}(R).
\end{align}
It is shown that
\begin{gather*}
    C^{l,m}_{l_1,m_1,l_2,m_2} = 0,\quad \text{ if } m \neq m_1 + m_2,\\
    \sum_{m_1,m_2} C^{l,m}_{l_1,m_1,l_2,m_2} C^{l',m'}_{l_1,m_1,l_2,m_2} = \delta_{l,l'}\delta_{m,m'},\\
    \sum_{l,m} C^{l,m}_{l_1,m_1,l_2,m_2} C^{l,m}_{l_1,m_1',l_2,m_2'} = \delta_{m_1,m_1'}\delta_{m_2,m_2'},\\
    C^{l,m}_{l_1,m_1,l_2,m_2} = (-1)^{l_1+l_2-l} C^{l,-m}_{l_1,-m_1,l_2,-m_2}.
\end{gather*}
\begin{align}
    D^{l_1}_{m_1,n_1} (R) D^{l_2}_{m_2,n_2}(R) = \sum_{l=\max\{|l_1-l_2|,|m_1+m_2|,|n_1+n_2|\}}^{l_1+l_2} C^{l,m_1+m_2}_{l_1,m_1,l_2,m_2} C^{l,n_1+n_2}_{l_1,n_1,l_2,n_2} D^{l}_{m_1+m_2,n_1+n_2}(R).
\end{align}
The coefficients $C^{l,m}_{l_1,m_1,l_2,m_2}$ can be arranged into a matrix~\cite{MarPec11}, namely $C_{l_1,l_2}$. 
The column index is determined by $(l,m)$. 
Since $|l_1-l_2|\leq l \leq l_1+l_2$ and $-l\leq m \leq l$, the number of columns is given by
\[
    \sum_{l=|l_1-l_2|}^{l_1+l_2} (2l+1) = (2l_1+1)(2l_2+1).
\]
The row index is determined by $(l_1,m_1,l_2,m_2)$. 
Since $-l_1\leq m_1 \leq l_1$ and $-l_2\leq m_2\leq l_2$, the number of rows is also $(2l_1+1)(2l_2+1)$. 
Therefore, $C_{l_1,l_2} \in \Re^{(2l_1+1)(2l_2+1)\times(2l_1+1)(2l_2+1)}$.
The ordering schemes are proposed in~\cite{MarPec11}, and their analytic forms are presented in~\cite{Dif17} as
\begin{align*}
    (\text{column index of } C^{l,m}_{l_1,m_1,l_2,m_2}) &= l^2-(l_2-l_1)^2 + l + m,\\
    (\text{row index of } C^{l,m}_{l_1,m_1,l_2,m_2}) &= (l_1+m_1)(2l_2+1) + l_2+m_2.
\end{align*}
In this formulation, 
\begin{equation}
    D^{l_1}(R) \otimes D^{l_2}(R) = C_{l_1,l_2} \bracket{ \bigoplus_{l=|l_1-l_2|}^{l_1+l_2} D^l(R)} C_{l_1,l_2}^T.\label{eqn:Clebsch_Gordon}
\end{equation}
A computational scheme is proposed in~\cite{Str14}. 

\section{Real Harmonic Analysis}

Next, we consider harmonic analysis for real-valued functions on $\SO$. 
For $D^l_{m,n}(R)$, the indices for $(m,n)$ belongs to the set $I^l=\{-l\leq m,n \leq l\}$. Let $I^l_-=\{(m,n)\in I_l\,|\, (m<0), \text{ or } (m=0,n<0)\}$, and $I^l_+=\{(m,n)\in I_l\,|\, (m>0), \text{ or } (m=0,n>0)\}$. Then, we can show
\[
I^l = I^l_- \cup \{(0,0)\} \cup I^l_+,
\]
and $I^l_-=\{(-m,-n)\,|\, (m,n)\in I^l_+\}$. Using these, the summation in the band-limited function given by \refeqn{fB} can be decomposed into
\begin{align*}
f(R(\alpha,\beta,\gamma)) &= \sum_{l=0}^{B-1} (2l+1) \tilde f^l_{0,0} D^l_{0,0}(\alpha,\beta,\gamma)\\
&\quad + \sum_{l=0}^{B-1} (2l+1) \sum_{(m,n)\in I^l_-}  \tilde f^l_{m,n} D^l_{m,n}(\alpha,\beta,\gamma)
+\tilde f^l_{-m,-n} D^l_{-m,-n}(\alpha,\beta,\gamma).
\end{align*}
From \refeqn{Dlmn}, \refeqn{Dlmn_bar}, 
\begin{align*}
f(R(\alpha,\beta,\gamma)) &= \sum_{l=0}^{B-1} (2l+1) \tilde f^l_{0,0} d^l_{0,0}(\alpha,\beta,\gamma)\\
&\quad + \sum_{l=0}^{B-1} (2l+1) \sum_{(m,n)\in I^l_-}  \tilde f^l_{m,n} D^l_{m,n}(\alpha,\beta,\gamma)
+(-1)^{m-n} \tilde f^l_{-m,-n} \overline{D^l_{m,n}(\alpha,\beta,\gamma)}.
\end{align*}
Consequently,  $f$ is real-valued iff the Fourier parameters satisfy
\begin{gather}
\overline{f^l_{0,0}}=f^l_{0,0},\\
\overline{\tilde f^l_{m,n}} = (-1)^{m-n} \tilde f^l_{-m,-n}.
\end{gather}
Under this constraints, the Fourier series is rewritten as
\begin{align*}
f(R(\alpha,\beta,\gamma)) &= \sum_{l=0}^{B-1} (2l+1) \tilde f^l_{0,0} d^l_{0,0}(\alpha,\beta,\gamma)\\
&\quad + \sum_{l=0}^{B-1} (2l+1) 2\sum_{(m,n)\in I^l_-}  
\mathrm{Re}[\tilde f^l_{m,n}] \mathrm{Re}[D^l_{m,n}(\alpha,\beta,\gamma)]
-\mathrm{Im}[\tilde f^l_{m,n}] \mathrm{Im}[D^l_{m,n}(\alpha,\beta,\gamma)].
\end{align*}
Therefore, any real-valued, band-limited function is expanded by $D^l_{m,n}$ for $(m,n)\in I^l_- \cup \{(0,0)\}=I^l\setminus I^l_+$ with $0\leq l\leq B-1$. 

\paragraph{Spherical Harmonics}
Instead of using the real and the imaginary parts of $D^l_{m,n}$, it is desirable to construct a real-valued irreducible representation. 
We will induce it from real spherical harmonics as follows. 

Spherical harmonics $Y_{lm}(\theta,\phi)$ forms a basis of square-integrable function on $\Sph^2$, and it is defined as
\begin{align}
    Y_{l,m}(\theta,\phi) = e^{im\phi} \sqrt{\frac{2l+1}{4\pi}\frac{(l-m)!}{(l+m)!}}  P^m_l(\cos\theta),
\end{align}
with $\theta\in[0,\pi]$ and $\phi\in[0,2\pi)$ are the polar coordinates of $\Sph^2$, parameterizing $x\in\Sph^2$ as 
\[
    x(\theta,\phi)=[\cos\phi\sin\theta, \sin\phi\sin\theta, \cos\theta]. 
\]
Due to Sturm-Liouville theory, it is given by the product of the $\phi$-dependent term and the $\theta$-dependent term. 
The former is given by the basis of $\Sph^1$ and the latter is given by the associated Legendre polynomials. 

Let $dx=\frac{1}{4\pi} \sin\theta d\phi d\theta$ be the surface area of $\Sph^2$, normalized such that $\int_{\Sph^2} dx =1$. 
We define an inner product on $\mathcal{L}^2(\Sph^2)$ as
\begin{align*}
    \pair{ f(x), g(x) } = \int_{\Sph^2} \overline{f(x)}g(x) dx.
\end{align*}
The spherical harmonics satisfies the following orthogonality,
\begin{align*}
    \pair{ Y_{l_1,m_1}(x), Y_{l_2,m_2}(x)} = \frac{1}{4\pi} \delta_{l_1l_2} \delta_{m_1m_2}.
\end{align*}
Also,
\[
    \overline{Y_{l,m}(x)} = (-1)^m Y_{l,-m}(x).
\]

Spherical harmonics is closely related to the wigner-D function. 
Interestingly, $R(\alpha,\beta,\gamma)e_3=x(\theta,\phi)$ with $\theta=\beta$ and $\phi=\alpha$: when the 3-2-3 Euler angles are used to parameterize $R\in\SO$, the third column of $R$ corresponds to the parameterization of $x\in\Sph^2$ with the polar coordinates.  
We can further show
\begin{align*}
    D^l_{m,0}(\alpha,\beta,\gamma) &= \sqrt{\frac{4\pi}{2l+1}}\overline{Y_{l,m}(\beta,\alpha)},\\
    D^l_{0,m}(\alpha,\beta,\gamma) &= (-1)^m\sqrt{\frac{4\pi}{2l+1}}\overline{Y_{l,m}(\beta,\gamma)}.
\end{align*}
Using the homomorphism property of the group representation, the above equations imply (see ~\cite[pp. 342]{ChiKya01})
\begin{align*}
    Y_{l,n}(R^T x) = \sum_{m'} Y_{l,m'}(x) D^l_{m',n}(R) .
\end{align*}

Therefore, the wigner-D function can be rediscovered from the spherical harmonics as
\begin{align*}
    \pair{Y_{l,m}(x), Y_{l,n}(R^T x) } &= \sum_{m'} \pair{Y_{l,m}(x), Y_{l,m'}(x) D^l_{m',n}(R) }\\
    & = \frac{1}{4\pi} D^l_{m,n}(R).
\end{align*}
Let $Y_l$ be the $2l+1$-by-one column whose elements are composed of $Y_{l,m}$ for $m\in\{-l,\ldots l\}$ in the ascending order. 
The above equation is rewritten in a matrix form as
\[
    \pair{ Y_l(x), (Y_l(R^Tx))^T } = \frac{1}{4\pi} D^l(R).
\]

\paragraph{Real Spherical Harmonics}

When spherical harmonics are considered as a basis for real-valued square-integrable functions on $\Sph^2$, real spherical harmonics can be formulated. 
In particular, real spherical harmonics $S_{l,m}$ is constructed by the following transform in~\cite{BlaFloJMS97}
\begin{align}
    S_l = T^l Y_l,\label{eqn:YtoS}
\end{align}
where $S_l$ is the $2l+1$-by-one column whose elements are composed of $S_{l,m}$ for $m\in\{-l,\ldots l\}$ in the ascending order. 
The matrix $T^l\in\Cp^{(2l+1)\times(2l+1)}$ is defined as
\begin{align*}
    T^l=\frac{1}{\sqrt{2}}
    \begin{bmatrix}
        i & 0 & \cdots & 0 & \cdots & 0 & -i(-1)^l\\
        0 & i & \cdots & 0 & \cdots & -i(-1)^{l-1} & 0 \\
        \vdots & \vdots & \ddots &\vdots & \iddots & \vdots & \vdots\\
        0 & 0 & \hdots & \sqrt{2} & \hdots & 0 & 0 \\
        \vdots & \vdots & \iddots &\vdots & \ddots & \vdots & \vdots\\
        0 & 1 & \cdots & 0 & \cdots & (-1)^{l-1} & 0 \\
        1 & 0 & \cdots & 0 & \cdots & 0 & (-1)^l\\
    \end{bmatrix}.
\end{align*}
Or equivalently, the elements of $T^l$ satisfy the following rules.
\begin{itemize}
    \item $T^l_{0,0}=1$
    \item $T^l_{m,n}=0$ if $|m|\neq|n|$
    \item $T^l_{-m,-m} = \frac{i}{\sqrt{2}}$ for $m>0$
    \item $T^l_{-m,m}=\frac{-i(-1)^m}{\sqrt{2}}$ for $m>0$
    \item $T^l_{m,-m}=\frac{1}{\sqrt{2}}$ for $m>0$
    \item $T^l_{m,m}=\frac{(-1)^m}{\sqrt{2}} $ for $m>0$
\end{itemize}
It is straightforward to show that the columns of $T^l$ are mutually orthonormal. 
Therefore,
\[
    (T^l)^{-1} = (T^l)^* = (\overline{T^l})^T,\quad T (\overline T)^T= \overline{T} T^T = I_{2l+1}.
\]

\paragraph{Real Harmonic Analysis on $\SO$}
Let $U^l_{m,n}(R)$ be the matrix representation for  real-valued square-integrable functions on $\SO$. 
Motivated by above~\cite{BlaFloJMS97}, it is defined as
\[
    \pair{ S_l(x), (S_l(R^Tx))^T } = \frac{1}{4\pi} U^l(R).
\]
From \eqref{eqn:YtoS},
\[
\frac{1}{4\pi} U^l(R) = \pair{ T^l Y_l(x),  (Y_l(R^T x))^T (T^l)^T } 
= \overline{T^l} \pair{ Y_l(x), (Y_1(R^T x))^T} (T^l)^T
= \frac{1}{4\pi} \overline{T^l} D^l(R) (T^l)^T,
\]
which yields the following real matrix representation of $SO$
\begin{equation}
    U^l (R) = \overline{T^l} D^l(R) (T^l)^T.\label{eqn:Ul}
\end{equation}
We can show that it is also a homomorphism as
\[
    U^l( R_1 R_2) = \overline{T^l} D^l(R_1)D^l(R_2)  (T^l)^T
    =\overline{T^l} D^l(R_1) (T^l)^T  \overline{T^l} D^l(R_2)  (T^l)^T
    = U^l(R_1) U^l(R_2).
\]

While $U^l(R)$ can be evaluated by transforming $D^l(R)$ according to \eqref{eqn:Ul}, the procedure will unnecessary steps involving complex variables. 
Therefore, we find an alternative expressions for each real element of $U^l$. 
Using element-wise operation of \eqref{eqn:Ul},
\begin{align}
    U^l_{m,n}(R) & = \sum_{p,q} \overline{T}^l_{m,p} T^l_{n,q} D^l_{p,q} (R).
\end{align}
According to the values of $m,n$, there are four cases of the above equation.
\begin{align*}
    U^l_{0,0} (R) &= D^l_{0,0} (R) = d^l_{0,0}(R),\\
    U^l_{m,0} (R) &= \overline{T}^l_{m,m} D^l_{m,0} (R) + \overline{T}^l_{m,-m} D^l_{-m,0} (R),\\
    U^l_{0,n} (R) &= T^l_{n,n} D^l_{0,n}(R) + T^l_{n,-n} D^l_{0,-n}(R),\\
    U^l_{m,n} (R) &= \overline{T}^l_{m,m} T^l_{n,n} D^l_{m,n} (R) + \overline{T}^l_{m,m} T^l_{n,-n} D^l_{m,-n} (R) + \overline{T}^l_{m,-m} T^l_{n,n} D^l_{-m,n} (R) + \overline{T}^l_{m,-m} T^l_{n,-n} D^l_{-m,-n} (R).
\end{align*}
For positive indices $m,n>0$, the last three cases are further divided into the following eight cases.
\begin{align*}
    U^l_{m,0} (R) &= \overline{T}^l_{m,m} D^l_{m,0} (R) + \overline{T}^l_{m,-m} D^l_{-m,0} (R),\\
    U^l_{-m,0} (R) &= \overline{T}^l_{-m,-m} D^l_{-m,0} (R) + \overline{T}^l_{-m,m} D^l_{m,0} (R),\\
    U^l_{0,n} (R) &= T^l_{n,n} D^l_{0,n}(R) + T^l_{n,-n} D^l_{0,-n}(R),\\
    U^l_{0,-n} (R) &= T^l_{-n,-n} D^l_{0,-n}(R) + T^l_{-n,n} D^l_{0,n}(R),\\
    U^l_{m,n} (R) &= \overline{T}^l_{m,m} T^l_{n,n} D^l_{m,n} (R) + \overline{T}^l_{m,m} T^l_{n,-n} D^l_{m,-n} (R) + \overline{T}^l_{m,-m} T^l_{n,n} D^l_{-m,n} (R) + \overline{T}^l_{m,-m} T^l_{n,-n} D^l_{-m,-n} (R),\\
    U^l_{m,-n} (R) &= \overline{T}^l_{m,m} T^l_{-n,-n} D^l_{m,-n} (R) + \overline{T}^l_{m,m} T^l_{-n,n} D^l_{m,n} (R) + \overline{T}^l_{m,-m} T^l_{-n,-n} D^l_{-m,-n} (R) + \overline{T}^l_{m,-m} T^l_{-n,n} D^l_{-m,n} (R),\\
    U^l_{-m,n} (R) &= \overline{T}^l_{-m,-m} T^l_{n,n} D^l_{-m,n} (R) + \overline{T}^l_{-m,-m} T^l_{n,-n} D^l_{-m,-n} (R) + \overline{T}^l_{-m,m} T^l_{n,n} D^l_{m,n} (R) + \overline{T}^l_{-m,m} T^l_{n,-n} D^l_{m,-n} (R),\\
    U^l_{-m,-n} (R) &= \overline{T}^l_{-m,-m} T^l_{-n,-n} D^l_{-m,-n} (R) + \overline{T}^l_{-m,-m} T^l_{-n,n} D^l_{-m,n} (R) + \overline{T}^l_{-m,m} T^l_{-n,-n} D^l_{m,-n} (R) + \overline{T}^l_{-m,m} T^l_{-n,n} D^l_{m,n} (R).
\end{align*}
Substituting the values of $T^l$, for $m,n>0$,
\begin{align*}
    U^l_{m,0} (R) &= \frac{(-1)^m}{\sqrt{2}} D^l_{m,0} (R) + \frac{1}{\sqrt{2}} D^l_{-m,0} (R),\\
    U^l_{-m,0} (R) &= \frac{-i}{\sqrt{2}} D^l_{-m,0} (R) + \frac{i(-1)^m}{\sqrt{2}} D^l_{m,0} (R),\\
    U^l_{0,n} (R) &= \frac{(-1)^n}{\sqrt{2}} D^l_{0,n}(R) + \frac{1}{\sqrt{2}} D^l_{0,-n}(R),\\
    U^l_{0,-n} (R) &= \frac{i}{\sqrt{2}} D^l_{0,-n}(R) + \frac{-i(-1)^n}{\sqrt{2}} D^l_{0,n}(R),\\
    U^l_{m,n} (R) &= \frac{(-1)^m}{\sqrt{2}} \frac{(-1)^n}{\sqrt{2}} D^l_{m,n} (R) + \frac{(-1)^m}{\sqrt{2}} \frac{1}{\sqrt{2}} D^l_{m,-n} (R) + \frac{1}{\sqrt{2}} \frac{(-1)^n}{\sqrt{2}} D^l_{-m,n} (R) + \frac{1}{\sqrt{2}} \frac{1}{\sqrt{2}} D^l_{-m,-n} (R),\\
    U^l_{m,-n} (R) &= \frac{(-1)^m}{\sqrt{2}} \frac{i}{\sqrt{2}} D^l_{m,-n} (R) + \frac{(-1)^m}{\sqrt{2}} \frac{-i(-1)^n}{\sqrt{2}} D^l_{m,n} (R) + \frac{1}{\sqrt{2}} \frac{i}{\sqrt{2}} D^l_{-m,-n} (R) + \frac{1}{\sqrt{2}} \frac{-i(-1)^n}{\sqrt{2}} D^l_{-m,n} (R),\\
    U^l_{-m,n} (R) &= \frac{-i}{\sqrt{2}} \frac{(-1)^n}{\sqrt{2}} D^l_{-m,n} (R) + \frac{-i}{\sqrt{2}} \frac{1}{\sqrt{2}} D^l_{-m,-n} (R) + \frac{i(-1)^m}{\sqrt{2}} \frac{(-1)^n}{\sqrt{2}} D^l_{m,n} (R) + \frac{i(-1)^m}{\sqrt{2}} \frac{1}{\sqrt{2}} D^l_{m,-n} (R),\\
    U^l_{-m,-n} (R) &= \frac{-i}{\sqrt{2}} \frac{i}{\sqrt{2}} D^l_{-m,-n} (R) + \frac{-i}{\sqrt{2}} \frac{-i(-1)^n}{\sqrt{2}} D^l_{-m,n} (R) + \frac{i(-1)^m}{\sqrt{2}} \frac{i}{\sqrt{2}} D^l_{m,-n} (R) + \frac{i(-1)^m}{\sqrt{2}} \frac{-i(-1)^n}{\sqrt{2}} D^l_{m,n} (R).
\end{align*}
The last four equations are rearranged into
\begin{align*}
    U^l_{m,n} (R) &= \frac{(-1)^{m+n}}{2}  D^l_{m,n} (R) + \frac{(-1)^m}{2}  D^l_{m,-n} (R) +  \frac{(-1)^n}{2} D^l_{-m,n} (R) + \frac{1}{2}  D^l_{-m,-n} (R),\\
    U^l_{m,-n} (R) &= \frac{i(-1)^m}{2}  D^l_{m,-n} (R) + \frac{-i(-1)^{m+n}}{2} D^l_{m,n} (R) +  \frac{i}{2} D^l_{-m,-n} (R) +  \frac{-i(-1)^n}{2} D^l_{-m,n} (R),\\
    U^l_{-m,n} (R) &=  \frac{-i(-1)^n}{2} D^l_{-m,n} (R) + \frac{-i}{2}  D^l_{-m,-n} (R) + \frac{i(-1)^{m+n}}{2}  D^l_{m,n} (R) + \frac{i(-1)^m}{2}  D^l_{m,-n} (R),\\
    U^l_{-m,-n} (R) &= \frac{1}{2}  D^l_{-m,-n} (R) +  \frac{-(-1)^n}{2} D^l_{-m,n} (R) + \frac{-(-1)^m}{2}  D^l_{m,-n} (R) + \frac{(-1)^{m+n}}{2}  D^l_{m,n} (R).
\end{align*}

Using the symmetry of the wigner-$d$ functions, and after rearranging, we obtain 
\begin{align}
    U^l_{m,0} (R) &= (-1)^m \sqrt{2} d^l_{m,0}(\beta) \cos m\alpha,\label{eqn:U_lm0} \\
    U^l_{-m,0} (R) &= \sqrt{2} d^l_{-m,0}(\beta) \sin m\alpha,\\
    U^l_{0,n} (R) &= (-1)^{n} \sqrt{2} d^l_{0,n}(\beta) \cos n\gamma, \\ 
    U^l_{0,-n} (R) &= -\sqrt{2} d^l_{0,-n}(\beta) \sin n\gamma,\\
    U^l_{m,n} (R) &= (-1)^{m+n} d^l_{m,n}(\beta) \cos (m\alpha + n\gamma) + (-1)^m d^l_{m,-n}(\beta) \cos(m\alpha - n\gamma),\\
    U^l_{m,-n} (R) & = (-1)^{m} d^l_{m,-n}(\beta) \sin(m\alpha-n\gamma) - (-1)^{m+n} d^l_{m,n}(\beta)\sin(m\alpha+n\gamma),\\
    U^l_{-m,n} (R) & = -(-1)^n d^l_{-m,n}(\beta) \sin(-m\alpha+n\gamma) + d^l_{-m,-n}(\beta) \sin(m\alpha+n\gamma),\\
    U^l_{-m,-n} (R) & = d^l_{-m,-n}(\beta) \cos (m\alpha + n\gamma) - (-1)^n d^l_{-m,n}(\beta) \cos(-m\alpha + n\gamma).\label{eqn:U_lmmmn}
\end{align} 
For arbitrary $m,n\in\{-l,-1+1,\ldots,l\}$, the above equations can be written as 
\begin{align*}
    U^l_{m,n} =  d^l_{|m|,|n|}(\beta) \Phi^1_{m,n}(\alpha,\gamma) + d^l_{|m|,-|n|}(\beta) \Phi^2_{m,n}(\alpha,\gamma),
\end{align*}
where
\begin{align*}
    \Phi^1_{m,n}(\alpha,\gamma) 
    & =\begin{cases}
    1 \quad &(m=0,n=0)\\
    (-1)^{m-n}\sqrt{2} \cos(m\alpha+n\gamma) \quad & (m>0,n=0) \text{ or } (m=0,n>0)\\
    -(-1)^{m-n}\sqrt{2} \sin(m\alpha-n\gamma) \quad & (m<0,n=0) \text{ or } (m=0,n<0)\\
    (-1)^{m-n} \cos(m\alpha+n\gamma) \quad & mn >0 \\
    -(-1)^{m-n} \sin(m\alpha-n\gamma) \quad & mn < 0 
\end{cases}\\
\Phi^2_{m,n}(\alpha,\gamma) 
    & =\begin{cases}
(-1)^{m}\mathrm{sgn}(m) \cos(m\alpha-n\gamma) \quad & mn >0 \\
(-1)^{m}\mathrm{sgn}(m) \sin(m\alpha+n\gamma) \quad & mn <0\\
0 & \text{otherwise}
    \end{cases}
\end{align*}
Or equivalently,
\begin{align*}
    U^l_{m,n} (R) =
    \begin{cases}
        \sin m\alpha \sin n\gamma \Theta^l_{m,n} (\beta) + \cos m\alpha \cos n\gamma \Psi^l_{m,n}(\beta) & (m\geq 0, n\geq 0) \text{ or } (m<0,n<0)\\
        \sin m\alpha \cos n\gamma \Theta^l_{m,n} (\beta) + \cos m\alpha \sin n\gamma \Psi^l_{m,n}(\beta) & (m\geq 0, n < 0) \text{ or } (m<0,n\geq0),
    \end{cases}
\end{align*}
where
\begin{align*}
    \Theta^l_{m,n}(\beta) & =
    \begin{cases}
        -(-1)^{m-n} d^l_{|m|,|n|}(\beta) + (-1)^m\mathrm{sgn}(m) d^l_{|m|,-|n|}(\beta) & mn\neq 0\\
        -(-1)^{m-n} \sqrt{2} d^l_{|m|,|n|}(\beta) & mn = 0 \text{ and } m^2+n^2 \neq 0\\
        0 & m=n=0.
    \end{cases}\\
    \Psi^l_{m,n}(\beta) & = 
    \begin{cases}
        (-1)^{m-n} d^l_{|m|,|n|}(\beta) + (-1)^m\mathrm{sgn}(m) d^l_{|m|,-|n|}(\beta) & mn\neq 0\\
        (-1)^{m-n} \sqrt{2} d^l_{|m|,|n|}(\beta) & mn = 0 \text{ and } m^2+n^2 \neq 0\\
        d^l_{0,0} & m=n=0.
    \end{cases}
\end{align*}
There are the symmetries of
\begin{align*}
    \Theta^l_{m,n}=\Theta^l_{m,-n},\quad
    \Psi^l_{m,n}=\Psi^l_{m,-n},\quad
    \Theta^l_{m,0}=\Theta^l_{-m,0},\quad
    \Psi^l_{m,0}=\Psi^l_{-m,0}.
\end{align*}



Numerical tests show that \eqref{eqn:U_lm0}--\eqref{eqn:U_lmmmn} are more computationally efficient than the above formulation involving $\Phi^1$ and $\Phi^2$. 

The orthogonality can be shown as follows. 
For any nonzero $m_1,n_1,m_2,n_2\in\mathbb{Z}$,
\[
    \int_{0}^{2\pi}\int_0^{2\pi} \cos(m_1\alpha+n_1\gamma)\cos(m_2\alpha+n_2\gamma) d\alpha d\gamma = 
    \begin{cases}
        2\pi^2 & (m_1=m_2,n_1=n_2) \text{ or } (m_1=-m_2,n_2=-n_2)\\
        0 & \text{otherwise}
    \end{cases}
\]
For $m_1,n_1,m_2,n_2 > 0$, 
\begin{align*}
    \pair{ U^{l_1}_{m_1,0}(R), U^{l_2}_{m_2,0}(R) } & = 2 \pair{ d^{l_1}_{m_1,0}(\beta) \cos m_1\alpha , d^{l_2}_{m_2,0}(\beta)\cos m_2\alpha },\\
                                                    & = \frac{1}{8\pi^2}\times  2 \times \delta_{m_1,m_2} \pi \times 2\pi \times \frac{2}{2l_1+1} \delta_{l_1,l_2}\\
                                                    & = \frac{1}{2l_1+1} \delta_{l_1,l_2}\delta{m_1,m_2}.
\end{align*}
Also,
\begin{align*}
    \pair{ U^{l_1}_{m_1,n_1}(R), U^{l_2}_{m_2,n_2}(R) } & =
                                                  \pair{ d^{l_1}_{m_1,n_1}(\beta)\cos(m_1\alpha+n_1\gamma), d^{l_2}_{m_2,n_2}(\beta)\cos(m_2\alpha+n_2\gamma) }\\
                                                  &\quad +\pair{ d^{l_1}_{m_1,n_1}(\beta)\cos(m_1\alpha+n_1\gamma), (-1)^n d^{l_2}_{m_2,-n_2}(\beta)\cos(m_2\alpha-n_2\gamma) }\\
                                                  &\quad +\pair{ d^{l_1}_{m_1,-n_1}(\beta)\cos(m_1\alpha-n_1\gamma), (-1)^nd^{l_2}_{m_2,n_2}(\beta)\cos(m_2\alpha+n_2\gamma) }\\
                                                  &\quad +\pair{ d^{l_1}_{m_1,-n_1}(\beta)\cos(m_1\alpha-n_1\gamma), d^{l_2}_{m_2,-n_2}(\beta)\cos(m_2\alpha-n_2\gamma) }\\
                                                  & = \frac{1}{8\pi^2} \delta_{l_1,l_2}\delta_{m_1,m_2} \delta_{n_1,n_2} \frac{2}{2l_1+1}\times 2\pi^2 \times 2\\
                                                  & = \frac{1}{2l_1+1} \delta_{l_1,l_2}\delta_{m_1,m_2} \delta_{n_1,n_2}.
\end{align*}
These can be repeated for all of other cases to show that
\begin{align}
    \pair{ U^{l_1}_{m_1,n_1}(R) , U^{l_2}_{m_2,n_2} (R) } =  \frac{1}{2l_1+1} \delta_{l_1,l_2}\delta_{m_1,m_2} \delta_{n_1,n_2}.\label{eqn:U_ortho}
\end{align}

\subsection{Character}

The character of $U^l(R)$ is identical to that of $D^l(R)$, as
\[
    \chi(R) = \trs{U^l(R)} = \trs{ \overline{C^l} D^l(R) (C^l)^T } = \trs{D^l(R) (C^l)^T \overline{C^l}} = \trs{D^l(R)}.
\]

\subsection{Operational Properties}
\subsection{Fourier Transform}
According to the Peter-Weyl theorem, any $f\in\mathcal{L}^2(\SO)$ has the following decomposition
\begin{align}
f(R(\alpha,\beta,\gamma)) &= \sum_{l=0}^\infty \sum_{m,n=-l}^l (2l+1)\tilde f^l_{m,n} U^l_{m,n}(\alpha,\beta,\gamma)\nonumber\\
                          &= \sum_{l=0}^\infty (2l+1)\trs{(\bar f^l)^T U^l(\alpha,\beta,\gamma)},\label{eqn:f_IFT_real}
\end{align}
which is the Fourier transform. 
From the orthogonality property \refeqn{U_ortho}, the Fourier parameter $\tilde f^l_{m,n}\in\Re$ is obtained by
\begin{align}
    \tilde f^l_{m,n} &= \pair{ f(R), U^l_{m,n}(R)}\nonumber\\
                     & =\frac{1}{8\pi^2}\int_0^{2\pi}\int_{0}^\pi\int_0^{2\pi} f(R(\alpha,\beta,\gamma)) {U^{l}_{m,n}(R(\alpha,\beta,\gamma))}d\alpha d\beta d\gamma,\label{eqn:f_FT_real}
\end{align}
which is the inverse transform. 

We also have the Plancherel theorem, stating
\[
\pair{f_1(R), f_2(R)} = \sum_{l=0}^\infty (2l+1)\pair{\bar f_1^l, \bar f_2^l},
\]
with the inner product $\pair{A,B}=\trs{A^TB}$ on $A,B\in\Re^{n\times n}$. Also, 
\[
\| f(R)\|^2 = \sum_{l=0}^\infty (2l+1)\|\bar f^l\|^2
\]

\subsection{Sampling}

A function $f\in\mathcal{L}^2(\SO)$ is called \textit{band-limited} with the band $B$ if $\tilde f^l=0$ for any $l\geq B$ in \refeqn{f_IFT_real}, or equivalently
\begin{align}
f(R(\alpha,\beta,\gamma)) &= \sum_{l=0}^{B-1} \sum_{m,n=-l}^l (2l+1)\tilde f^l_{m,n} U^l_{m,n}(\alpha,\beta,\gamma).\label{eqn:fB_real}
\end{align}
The classical sampling theorem stated that the Fourier transform of a band limited function can be recovered from the sample values of the function that are chosen at a uniform grid with with a frequency that is at least twice of the band limit. 
Consequently, the Fourier transform of the above function can be computed from finite samples. 

Consider the following uniform grid for $(\alpha,\beta,\gamma)\in[0,2\pi)\times[0,\pi]\times[0,2\pi)$:
\[
[\SO]_d=\{(\alpha_{j_1},\beta_k,\gamma_{j_2})\,|\, j_1,j_2,k\in\{0,\ldots, 2B-1\}\},
\]
with
\begin{equation}\label{eqn:grid_real}
\alpha_j=\gamma_j= \frac{\pi j}{B},\quad \beta_k = \frac{\pi (2k+1)}{4B}.
\end{equation}

Define a sampling distribution of bandwidth $B$ as
\begin{equation}
s(R(\alpha,\beta,\gamma))=\sum_{j_1,k,j_2=0}^{2B-1} w_k \delta_{R(\alpha,\beta,\gamma),R(\alpha_{j_1},\beta_k,\gamma_{j_2})},
\end{equation}
which is the linear combination of grid points weighted by the parameter $w_k$. 
The Fourier transform of $s$ is given by
\begin{align*}
\bar s^l_{m,n} & = \pair{s(R(\alpha,\beta,\gamma)), U^l_{m,n}(R)}\\
               & = \sum_{j_1,k,j_2=0}^{2B-1} w_k U^l_{m,n}(R(\alpha_{j_1},\beta_k,\gamma_{j_2}))\\
               & = \sum_{j_1,j_2=0}^{2B-1} \Phi^1_{m,n}(\alpha_{j_1},\gamma_{j_2}) \sum_{k=0}^{2B-1} w_k d^l_{|m|,|n|}(\beta_k)\\
               & \quad + \sum_{j_1,j_2=0}^{2B-1} \Phi^2_{m,n}(\alpha_{j_1},\gamma_{j_2}) \sum_{k=0}^{2B-1} w_k d^l_{|m|,-|n|}(\beta_k).
\end{align*}
For the selected grid, it is straightforward to show that
\[
    \sum_{j_1,j_2=0}^{2B-1} \Phi^1_{m,n}(\alpha_{j_1},\gamma_{j_2})=4B^2 \delta_{m,0}\delta_{n,0},\quad
    \sum_{j_1,j_2=0}^{2B-1} \Phi^2_{m,n}(\alpha_{j_1},\gamma_{j_2})=0.
\]
Therefore, it reduces to
\begin{align*}
\bar s^l_{m,n}  
& = 4B^2 \delta_{m,0}\delta_{n,0}\sum_{k=0}^{2B-1} w_k d^l_{0,0}(\beta_k),\\
& = 4B^2 \delta_{m,0}\delta_{n,0}\sum_{k=0}^{2B-1} w_k P_l(\beta_k),
\end{align*}
where $P_l$ is the $l$-th Legendre polynomial. 
We select the weight such that
\begin{equation}
\sum_{k=0}^{2B-1} w_k P_l(\beta_k) = \frac{1}{4B^2} \delta_{l,0},\quad l=0,\ldots 2B-1.
\end{equation}
In~\cite{DriHeaAAM94}, it is explicitly given as
\begin{equation}
w_k = \frac{1}{4B^3}\sin\beta_k \sum_{j=0}^{B-1} \frac{1}{2j+1}\sin((2j+1)\beta_k).
\end{equation}
Substituting this, the Fourier transform of the sampling distribution is given by
\begin{equation}
\bar s ^l_{m,n}=\delta_{m,0}\delta_{n,0}\delta_{l,0},\quad l=0,\ldots 2B-1.
\label{eqn:bar_s_real}
\end{equation}

Define $f_s(R)=f(R) s(R)$. 
It is straightforward to show
\begin{align*}
&f_s(R(\alpha,\beta,\gamma))\\
& = 
\sum_{j_1,k,j_2=0}^{2B-1} w_k f(R(\alpha_{j_1},\beta_k,\gamma_{j_2})) 
\delta_{R(\alpha,\beta,\gamma),R(\alpha_{j_1},\beta_k,\gamma_{j_2})},
\end{align*}
and the Fourier transform is given by
\begin{align*}
&(\bar{f_s})^l_{m,n}\\
& = 
\sum_{j_1,k,j_2=0}^{2B-1} w_k f(R(\alpha_{j_1},\beta_k,\gamma_{j_2})) 
{U^l_{m,n}(R(\alpha_{j_1},\beta_k,\gamma_{j_2}))}.
\end{align*}
Next, we show that this corresponds to the Fourier coefficient of $f$ within the band limit. 

From \refeqn{bar_s_real}, the sampling distribution can be expanded as
\[
s(R)=1+\sum_{l=2B}^\infty \sum_{m,n=-l}^l \tilde s^l_{m,n} U^l_{m,n}(R).
\]
Then,
\begin{align*}
    f_s(R)&=f(R)+f(R)\sum_{l=2B}^\infty \sum_{m,n=-l}^l \tilde s^l_{m,n} U^l_{m,n}(R).
\end{align*}
Since $f(R)$ can be expanded as a linear combination of $U^{l_1}$ for $0\leq l_1 \leq B-1$. 
The last term of the above equation is expanded by the product $U^{l_1}U^{l_2}$ with $0\leq l_1 \leq B-1$  and $2B\leq l_2$. 
According to the Clebsch-Gordon theorem, $U^{l_1}U^{l_2}$ is a linear combination of $U^{l_3}$ for $|l_1-l_2|\leq l_3 \leq l_1+l_2$. 
We have $\min|l_1-l_2|=2B-1-B=B+1$. 
As such, $f$ and $f_s$ share the Fourier coefficients in the given band limit. 
Or equivalently, 
\begin{align}
    \tilde f^l_{m,n}
    & = 
    \sum_{j_1,k,j_2=0}^{2B-1} w_k f(R(\alpha_{j_1},\beta_k,\gamma_{j_2})) {U^l_{m,n}(R(\alpha_{j_1},\beta_k,\gamma_{j_2}))}.
\end{align}

The above summation can be decomposed into
\begin{align*}
    \tilde f^l_{m,n} & = \sum_{j_1,j_2=0}^{2B-1} \Phi^1_{m,n}(\alpha_{j_1},\gamma_{j_2}) \left\{ \sum_{k=0}^{2B-1} w_k d^l_{|m|,|n|}(\beta_k) f(\alpha_{j_1},\beta_k,\gamma_{j_2}) \right\}\\
                     & \quad + \sum_{j_1,j_2=0}^{2B-1} \Phi^2_{m,n}(\alpha_{j_1},\gamma_{j_2}) \left\{ \sum_{k=0}^{2B-1} w_k d^l_{|m|,-|n|}(\beta_k) f(\alpha_{j_1},\beta_k,\gamma_{j_2}) \right\}.
\end{align*}



Or equivalently,
\begin{align*}
    U^l_{m,n} (R) =
    \begin{cases}
        \sin m\alpha \sin n\gamma \Theta^l_{m,n} (\beta) + \cos m\alpha \cos n\gamma \Psi^l_{m,n}(\beta) & (m\geq 0, n\geq 0) \text{ or } (m<0,n<0)\\
        \sin m\alpha \cos n\gamma \Theta^l_{m,n} (\beta) + \cos m\alpha \sin n\gamma \Psi^l_{m,n}(\beta) & (m\geq 0, n < 0) \text{ or } (m<0,n\geq0),
    \end{cases}
\end{align*}
This can be computed in the following order:
\begin{align*}
    F^{\theta,\beta}_{l,m,n}(j_1,j_2) &= \sum_{k=0}^{2B-1}w_k f(\alpha_{j_1},\beta_k,\gamma_{j_2})\Phi^l_{m,n}(\beta_k),\\
    F^{\psi,\beta}_{l,m,n}(j_1,j_2) &= \sum_{k=0}^{2B-1}w_k f(\alpha_{j_1},\beta_k,\gamma_{j_2}) \Theta^l_{m,n}(\beta_k),\\
F^{\theta,\gamma}_{l,m,n}(j_1)&=
\begin{cases}
    \sum_{j_2=0}^{2B-1} \sin n\gamma_{j_2} F^{\theta,\beta}_{l,m,n}(j_1,j_2) & (m\geq 0,n\geq 0) \text{ or } (m<0,n<0)\\
    \sum_{j_2=0}^{2B-1} \cos n\gamma_{j_2} F^{\theta,\beta}_{l,m,n}(j_1,j_2) & (m\geq 0, n<0) \text{ or } (m<0,n\geq 0)
\end{cases}\\
F^{\psi,\gamma}_{l,m,n}(j_1)&=
\begin{cases}
    \sum_{j_2=0}^{2B-1} \cos n\gamma_{j_2} F^{\psi,\beta}_{l,m,n}(j_1,j_2) & (m\geq 0,n\geq 0) \text{ or } (m<0,n<0)\\
    \sum_{j_2=0}^{2B-1} \sin n\gamma_{j_2} F^{\psi,\beta}_{l,m,n}(j_1,j_2) & (m\geq 0, n<0) \text{ or } (m<0,n\geq 0)
\end{cases}\\
f^l_{m,n} & = \sum_{j_1=0}^{2B-1} \sin m\alpha_{j_1} F^{\theta,\gamma}_{l,m,n}(j_1) + \cos m\alpha_{j_1} F^{\psi,\gamma}_{l,m,n}(j_1).
\end{align*}
where the last two operations are Fourier transforms in a linear space, which can be computed by various FFT algorithms. 

\subsection{Clebsch-Gordon Coefficients}

From \eqref{eqn:Ul},
\begin{align*}
    U^{l_1}(R) \otimes U^{l_2}(R) = (\overline{T}^{l_1} D^{l_1}(R) (T^{l_1})^T) \otimes (\overline{T}^{l_2} D^{l_2}(R) (T^{l_2})^T) .
\end{align*}
Repeatedly using the property of Kronecker product, $(AC)\otimes (BD) = (A\otimes B)(C \otimes D)$, 
\begin{align*}
    U^{l_1}(R) \otimes U^{l_2}(R) = (\overline{T}^{l_1} \otimes \overline{T}^{l_2}) (D^{l_1}(R) \otimes D^{l_2}(R)) (T^{l_1} \otimes   T^{l_2})^T .
\end{align*}
Substituting \eqref{eqn:Clebsch_Gordon} and \eqref{eqn:Ul},
\begin{align*}
    U^{l_1}(R) \otimes U^{l_2}(R) = (\overline{T}^{l_1} \otimes \overline{T}^{l_2}) C_{l_1,l_2} \bracket{ \bigoplus_{l=|l_1-l_2|}^{l_1+l_2} (T^l)^T U^l(R) \overline{T}^l } C_{l_1,l_2}^T (T^{l_1} \otimes   T^{l_2})^T .
\end{align*}
Define the Clebsch-Gordon matrix for the real harmonics as
\begin{align}
    c_{l_1,l_2} = (\overline{T}^{l_1} \otimes \overline{T}^{l_2}) C_{l_1,l_2} \bracket{ \bigoplus_{l=|l_1-l_2|}^{l_1+l_2} (T^l)^T }.
\end{align}
From the ordering rules of of $C_{l_1,l_2}$, the corresponding element of $c_{l_1,l_2}$ can be written as
\begin{align}
    c^{l,m}_{l_1,m_1,l_2,m_2} = \sum_{p_1=-l_1}^{l_1} \sum_{p_2 = -l_2}^{l_2} \sum_{p=-l}^l \overline T^{l_1}_{m_1,p_1} \overline T^{l_2}_{m_2,p_2} T^l_{m,p} C^{l,p}_{l_1,p_1,l_2,p_2}.
\end{align}
We have $T^l_{m,n}=0$ if $|m|\neq|n|$, and $C^{l,m}_{l_1,m_1,l_2,m_2}=0$ if $m\neq m_1+m_2$. 
Therefore, the summation in the above equation can be reduced to
\begin{align}
    c^{l,m}_{l_1,m_1,l_2,m_2} & = \sum_{p_1\in\{-m_1,m_1\}} \sum_{p_2 =\{ -m_2,m_2\}} \sum_{p=-l}^l \overline T^{l_1}_{m_1,p_1} \overline T^{l_2}_{m_2,p_2} T^l_{m,p} C^{l,p}_{l_1,p_1,l_2,p_2}\\
                              & = \sum_{p_1\in\{-m_1,m_1\}} \sum_{p_2 =\{ -m_2,m_2\}} \delta_{|m|,|p_1+p_2|} \overline T^{l_1}_{m_1,p_1} \overline T^{l_2}_{m_2,p_2} T^l_{m,p_1+p_2} C^{l,p_1+p_2}_{l_1,p_1,l_2,p_2}.
\end{align}
This implies
\begin{align}
    c^{l,m}_{l_1,m_1,l_2,m_2} = 0, \text{ if  $|m|\neq |m_1+m_2|$ or $|m|\neq |m_1-m_2|$}.\label{eqn:c_lm_zero}
\end{align}

We can expand the above expression for several cases.

\begin{itemize}
    \item $m_1=0$ and $m_2=0$
        $$c^{l,0}_{l_1,0,l_2,0} =  C^{l,0}_{l_1,0,l_2,0}$$

    \item $m_1\neq 0$ and $m_2 = 0$
        \begin{align*}
            c^{l,m_1}_{l_1,m_1,l_2,0} & =  \overline T^{l_1}_{m_1,m_1}  T^l_{m_1,m_1} C^{l,m_1}_{l_1,m_1,l_2,0} + \overline T^{l_1}_{m_1,-m_1}  T^l_{m_1,-m_1} C^{l,-m_1}_{l_1,-m_1,l_2,0} \\
                                      & = \frac{1}{2}(1+(-1)^{l_1+l_2-l}) C^{l,m_1}_{l_1,m_1,l_2,0},\\
            c^{l,-m_1}_{l_1,m_1,l_2,0} & =  \overline T^{l_1}_{m_1,-m_1}  T^l_{-m_1,-m_1} C^{l,-m_1}_{l_1,-m_1,l_2,0} + \overline T^{l_1}_{m_1,m_1}  T^l_{-m_1,m_1} C^{l,m_1}_{l_1,m_1,l_2,0} \\
                                      & = \frac{i}{2}(1-(-1)^{l_1+l_2-l}) C^{l,-m_1}_{l_1,-m_1,l_2,0}.
        \end{align*}

    \item $m_1 = 0$ and $m_2\neq 0$
        \begin{align*}
            c^{l,m_2}_{l_1,0,l_2,m_2} & =  \overline T^{l_2}_{m_2,m_2} T^l_{m_2,m_2} C^{l,m_2}_{l_1,0,l_2,m_2} +  \overline T^{l_2}_{m_2,-m_2} T^l_{m_2,-m_2} C^{l,-m_2}_{l_1,0,l_2,-m_2} ,\\
                                      & = \frac{1}{2}(1+(-1)^{l_1+l_2-l}) C^{l,m_2}_{l_1,0,l_2,m_2},\\
            c^{l,-m_2}_{l_1,0,l_2,m_2} & =  \overline T^{l_2}_{m_2,-m_2} T^l_{-m_2,-m_2} C^{l,-m_2}_{l_1,0,l_2,-m_2} +  \overline T^{l_2}_{m_2,m_2} T^l_{-m_2,m_2} C^{l,m_2}_{l_1,0,l_2,m_2} ,\\
                                      & = \frac{i}{2}(1-(-1)^{l_1+l_2-l}) C^{l,-m_2}_{l_1,0,l_2,-m_2}.
        \end{align*}

    \item $m_1 > 0$ and $m_2 > 0$
        \begin{align*}
            c^{l,m_1+m_2}_{l_1,m_1,l_2,m_2} & =  \overline T^{l_1}_{m_1,m_1} \overline T^{l_2}_{m_2,m_2} T^l_{m_1+m_2,m_1+m_2} C^{l,m_1+m_2}_{l_1,m_1,l_2,m_2}\\
                                            & \quad + \overline T^{l_1}_{m_1,-m_1} \overline T^{l_2}_{m_2,-m_2} T^l_{m_1+m_2,-m_1-m_2} C^{l,-m_1-m_2}_{l_1,-m_1,l_2,-m_2} ,\\
                                            & = \frac{1}{\sqrt{8}} ( 1 + (-1)^{l_1+l_2-l}) C^{l,m_1+m_2}_{l_1,m_1,l_2,m_2}.
        \end{align*}
        \begin{align*}
            c^{l,-m_1-m_2}_{l_1,m_1,l_2,m_2} & = \overline T^{l_1}_{m_1,-m_1} \overline T^{l_2}_{m_2,-m_2} T^l_{-m_1-m_2,-m_1-m_2} C^{l,-m_1-m_2}_{l_1,-m_1,l_2,-m_2} \nonumber\\
                                             &\quad + \overline T^{l_1}_{m_1,m_1} \overline T^{l_2}_{m_2,m_2} T^l_{-m_1-m_2,m_1+m_2} C^{l,m_1+m_2}_{l_1,m_1,l_2,m_2} ,\\ 
                                            & = \frac{i}{\sqrt{8}} ( 1 - (-1)^{l_1+l_2-l}) C^{l,-m_1-m_2}_{l_1,-m_1,l_2,-m_2}.
        \end{align*}

    \item $m_1 > 0 > m_2 $ and $m_1+m_2 > 0$
        \begin{align*}
            c^{l,m_1+m_2}_{l_1,m_1,l_2,m_2} & = \frac{i(-1)^{m_2}}{\sqrt{8}} ( -1 + (-1)^{l_1+l_2-l}) C^{l,m_1+m_2}_{l_1,m_1,l_2,m_2}.
        \end{align*}
        \begin{align*}
            c^{l,-m_1-m_2}_{l_1,m_1,l_2,m_2} & = \frac{-(-1)^{m_2}}{\sqrt{8}} ( 1 + (-1)^{l_1+l_2-l}) C^{l,-m_1-m_2}_{l_1,-m_1,l_2,-m_2}.
        \end{align*}

    \item $m_1 > 0 > m_2 $ and $m_1+m_2 = 0$
        \begin{align*}
            c^{l,m_1+m_2}_{l_1,m_1,l_2,m_2} & = \frac{i(-1)^{m_1}}{2} ( -1 + (-1)^{l_1+l_2-l}) C^{l,m_1+m_2}_{l_1,m_1,l_2,m_2}.
        \end{align*}
        \begin{align*}
            c^{l,-m_1-m_2}_{l_1,m_1,l_2,m_2} & = \frac{i(-1)^{m_2}}{2} ( 1 - (-1)^{l_1+l_2-l}) C^{l,-m_1-m_2}_{l_1,-m_1,l_2,-m_2}.
        \end{align*}

    \item $m_1 > 0 > m_2 $ and $m_1+m_2 < 0$
        \begin{align*}
            c^{l,m_1+m_2}_{l_1,m_1,l_2,m_2} & = \frac{(-1)^{m_1}}{\sqrt{8}} ( 1 + (-1)^{l_1+l_2-l}) C^{l,m_1+m_2}_{l_1,m_1,l_2,m_2}.
        \end{align*}
        \begin{align*}
            c^{l,-m_1-m_2}_{l_1,m_1,l_2,m_2} & = \frac{i(-1)^{m_1}}{\sqrt{8}} ( 1 - (-1)^{l_1+l_2-l}) C^{l,-m_1-m_2}_{l_1,-m_1,l_2,-m_2}.
        \end{align*}

    \item $m_2 > 0 > m_1 $ and $m_1+m_2 < 0$
        \begin{align*}
            c^{l,m_1+m_2}_{l_1,m_1,l_2,m_2} & = \frac{(-1)^{m_2}}{\sqrt{8}} ( 1 + (-1)^{l_1+l_2-l}) C^{l,m_1+m_2}_{l_1,m_1,l_2,m_2}.
        \end{align*}
        \begin{align*}
            c^{l,-m_1-m_2}_{l_1,m_1,l_2,m_2} & = \frac{i(-1)^{m_2}}{\sqrt{8}} ( 1 - (-1)^{l_1+l_2-l}) C^{l,-m_1-m_2}_{l_1,-m_1,l_2,-m_2}.
        \end{align*}

    \item $m_2 > 0 > m_1 $ and $m_1+m_2 = 0$
        \begin{align*}
            c^{l,m_1+m_2}_{l_1,m_1,l_2,m_2} & = \frac{i(-1)^{m_1}}{2} ( -1 + (-1)^{l_1+l_2-l}) C^{l,m_1+m_2}_{l_1,m_1,l_2,m_2}.
        \end{align*}
        \begin{align*}
            c^{l,-m_1-m_2}_{l_1,m_1,l_2,m_2} & = \frac{i(-1)^{m_1}}{2} ( 1 - (-1)^{l_1+l_2-l}) C^{l,-m_1-m_2}_{l_1,-m_1,l_2,-m_2}.
        \end{align*}

    \item $m_2 > 0 > m_1 $ and $m_1+m_2 > 0$
        \begin{align*}
            c^{l,m_1+m_2}_{l_1,m_1,l_2,m_2} & = \frac{i(-1)^{m_1}}{\sqrt{8}} ( -1 + (-1)^{l_1+l_2-l}) C^{l,m_1+m_2}_{l_1,m_1,l_2,m_2}.
        \end{align*}
        \begin{align*}
            c^{l,-m_1-m_2}_{l_1,m_1,l_2,m_2} & = \frac{-(-1)^{m_1}}{\sqrt{8}} ( 1 + (-1)^{l_1+l_2-l}) C^{l,-m_1-m_2}_{l_1,-m_1,l_2,-m_2}.
        \end{align*}

    \item $m_1 < 0$ and $m_2 < 0$
        \begin{align*}
            c^{l,m_1+m_2}_{l_1,m_1,l_2,m_2} & = \frac{i}{\sqrt{8}} ( -1 + (-1)^{l_1+l_2-l}) C^{l,m_1+m_2}_{l_1,m_1,l_2,m_2}.
        \end{align*}
        \begin{align*}
            c^{l,-m_1-m_2}_{l_1,m_1,l_2,m_2} & = \frac{-1}{\sqrt{8}} ( 1 + (-1)^{l_1+l_2-l}) C^{l,-m_1-m_2}_{l_1,-m_1,l_2,-m_2}.
        \end{align*}

    \item $m_1 > 0$ and $m_2 < 0$
        \begin{align*}
            c^{l,m_1-m_2}_{l_1,m_1,l_2,m_2} & =  \overline T^{l_1}_{m_1,m_1} \overline T^{l_2}_{m_2,-m_2} T^l_{m_1-m_2,m_1-m_2} C^{l,m_1-m_2}_{l_1,m_1,l_2,-m_2}\\
                                            & \quad + \overline T^{l_1}_{m_1,-m_1} \overline T^{l_2}_{m_2,m_2} T^l_{m_1-m_2,-m_1+m_2} C^{l,-m_1+m_2}_{l_1,-m_1,l_2,m_2} \\
                                            & = \frac{i}{\sqrt{8}} ( 1 - (-1)^{l_1+l_2-l}) C^{l,m_1-m_2}_{l_1,m_1,l_2,-m_2}.
        \end{align*}
        \begin{align*}
            c^{l,-m_1+m_2}_{l_1,m_1,l_2,m_2} & = \overline T^{l_1}_{m_1,-m_1} \overline T^{l_2}_{m_2,m_2} T^l_{-m_1+m_2,-m_1+m_2} C^{l,-m_1+m_2}_{l_1,-m_1,l_2,m_2} \nonumber\\
                                             &\quad + \overline T^{l_1}_{m_1,m_1} \overline T^{l_2}_{m_2,-m_2} T^l_{-m_1+m_2,m_1-m_2} C^{l,m_1-m_2}_{l_1,m_1,l_2,-m_2} \\
                                             & =\frac{1}{\sqrt{8}} ( 1 + (-1)^{l_1+l_2-l}) C^{l,-m_1+m_2}_{l_1,-m_1,l_2,m_2}.
        \end{align*}

    \item $m_1 < 0$ and $m_2 > 0$
        \begin{align*}
            c^{l,m_1-m_2}_{l_1,m_1,l_2,m_2} & = \frac{1}{\sqrt{8}} ( 1 + (-1)^{l_1+l_2-l}) C^{l,m_1-m_2}_{l_1,m_1,l_2,-m_2}.
        \end{align*}
        \begin{align*}
            c^{l,-m_1+m_2}_{l_1,m_1,l_2,m_2} & = \frac{i}{\sqrt{8}} ( 1 - (-1)^{l_1+l_2-l}) C^{l,-m_1+m_2}_{l_1,-m_1,l_2,m_2}.
        \end{align*}

    \item $m_1 > 0$ and $m_2 > 0$ and $m_1-m_2 >0$
        \begin{align*}
            c^{l,m_1-m_2}_{l_1,m_1,l_2,m_2} & = \frac{(-1)^{m_2}}{\sqrt{8}} ( 1 + (-1)^{l_1+l_2-l}) C^{l,m_1-m_2}_{l_1,m_1,l_2,-m_2}.
        \end{align*}
        \begin{align*}
            c^{l,-m_1+m_2}_{l_1,m_1,l_2,m_2} & = \frac{i(-1)^{m_2}}{\sqrt{8}} ( 1 - (-1)^{l_1+l_2-l}) C^{l,-m_1+m_2}_{l_1,-m_1,l_2,m_2}.
        \end{align*}

    \item $m_1 > 0$ and $m_2 > 0$ and $m_1-m_2 = 0$
        \begin{align*}
            c^{l,m_1-m_2}_{l_1,m_1,l_2,m_2} & = \frac{(-1)^{m_1}}{2} ( 1 + (-1)^{l_1+l_2-l}) C^{l,m_1-m_2}_{l_1,m_1,l_2,-m_2}.
        \end{align*}
        \begin{align*}
            c^{l,-m_1+m_2}_{l_1,m_1,l_2,m_2} & = \frac{(-1)^{m_1}}{2} ( 1 + (-1)^{l_1+l_2-l}) C^{l,-m_1+m_2}_{l_1,-m_1,l_2,m_2}.
        \end{align*}

    \item $m_1 > 0$ and $m_2 > 0$ and $m_1-m_2 < 0$
        \begin{align*}
            c^{l,m_1-m_2}_{l_1,m_1,l_2,m_2} & = \frac{i(-1)^{m_1}}{\sqrt{8}} ( 1 - (-1)^{l_1+l_2-l}) C^{l,m_1-m_2}_{l_1,m_1,l_2,-m_2}.
        \end{align*}
        \begin{align*}
            c^{l,-m_1+m_2}_{l_1,m_1,l_2,m_2} & = \frac{(-1)^{m_1}}{\sqrt{8}} ( 1 + (-1)^{l_1+l_2-l}) C^{l,-m_1+m_2}_{l_1,-m_1,l_2,m_2}.
        \end{align*}

    \item $m_1 < 0$ and $m_2 < 0$ and $m_1-m_2 >0$
        \begin{align*}
            c^{l,m_1-m_2}_{l_1,m_1,l_2,m_2} & = \frac{(-1)^{m_1}}{\sqrt{8}} ( 1 + (-1)^{l_1+l_2-l}) C^{l,m_1-m_2}_{l_1,m_1,l_2,-m_2}.
        \end{align*}
        \begin{align*}
            c^{l,-m_1+m_2}_{l_1,m_1,l_2,m_2} & = \frac{i(-1)^{m_1}}{\sqrt{8}} ( 1 - (-1)^{l_1+l_2-l}) C^{l,-m_1+m_2}_{l_1,-m_1,l_2,m_2}.
        \end{align*}

    \item $m_1 < 0$ and $m_2 < 0$ and $m_1-m_2 = 0$
        \begin{align*}
            c^{l,m_1-m_2}_{l_1,m_1,l_2,m_2} & = \frac{(-1)^{m_1}}{2} ( 1 + (-1)^{l_1+l_2-l}) C^{l,m_1-m_2}_{l_1,m_1,l_2,-m_2}.
        \end{align*}
        \begin{align*}
            c^{l,-m_1+m_2}_{l_1,m_1,l_2,m_2} & = \frac{(-1)^{m_1}}{2} ( 1 + (-1)^{l_1+l_2-l}) C^{l,-m_1+m_2}_{l_1,-m_1,l_2,m_2}.
        \end{align*}

    \item $m_1 < 0$ and $m_2 < 0$ and $m_1-m_2 < 0$
        \begin{align*}
            c^{l,m_1-m_2}_{l_1,m_1,l_2,m_2} & = \frac{i(-1)^{m_2}}{\sqrt{8}} ( 1 - (-1)^{l_1+l_2-l}) C^{l,m_1-m_2}_{l_1,m_1,l_2,-m_2}.
        \end{align*}
        \begin{align*}
            c^{l,-m_1+m_2}_{l_1,m_1,l_2,m_2} & = \frac{(-1)^{m_2}}{\sqrt{8}} ( 1 + (-1)^{l_1+l_2-l}) C^{l,-m_1+m_2}_{l_1,-m_1,l_2,m_2}.
        \end{align*}


\end{itemize}

\newcolumntype{C}{>{$\displaystyle} c <{$}}

\begin{table}
    \begin{center}
        \begin{threeparttable}
            % \renewcommand{\arraystretch}{3}
            \begin{tabular}{c | CCCC | C|C|C|C}\hline\hline
                Case & m_1 & m_2 & m_1+m_2 & m_1-m_2 & \dfrac{c^{l,m_1+m_2}_{l_1,m_1,l_2,m_2}}{ C^{m_1+m_2}_{l_1,m_1,l_2,m_2}}  & \dfrac{c^{l,-m_1-m_2}_{l_1,m_1,l_2,m_2}}{ C^{-m_1-m_2}_{l_1,m_1,l_2,m_2}}  & \dfrac{c^{l,m_1-m_2}_{l_1,m_1,l_2,m_2}}{ C^{m_1-m_2}_{l_1,m_1,l_2,m_2}}  & \dfrac{c^{l,-m_1+m_2}_{l_1,m_1,l_2,m_2}}{ C^{m_1-m_2}_{l_1,m_1,l_2,m_2}}  \\ \hline
                0 & 0 & 0 & 0 & 0 & 1 & 1 & 1 & 1\\\hline
                1 & + & 0 & + & + & \frac{1}{2}\eta & \frac{i}{2}\zeta & \\\hline
                2 & + & + & + & + & \multirow{3}{*}{$\frac{1}{\sqrt{8}}\eta$} & \multirow{3}{*}{$\frac{i}{\sqrt{8}}\zeta$} \\\cline{1-5}\cline{8-9}
                3 & + & + & + & 0 & & & &\\\cline{1-5}\cline{8-9}
                4 & + & + & + & - & & & & \\\hline
                5 & 0 & + & + & - & \frac{1}{2}\eta & \frac{i}{2}\zeta &\\\hline
                6 & - & + & + & - & -\frac{i(-1)^{m_1}}{\sqrt{8}}\zeta & -\frac{(-1)^{m_1}}{\sqrt{8}}\eta \\\hline
                7 & - & + & 0 & - & -\frac{i(-1)^{m_1}}{2}\zeta & \frac{i(-1)^{m_1}}{2}\zeta & \\\hline
                8 & - & + & - & - & \frac{(-1)^{m_2}}{\sqrt{8}}\eta & \frac{i(-1)^{m_2}}{\sqrt{8}}\zeta & \\\hline
                9 & - & 0 & - & - & \frac{1}{2}\eta & \frac{i}{2}\zeta \\\hline
                10 & - & - & - & - & \multirow{3}{*}{$-\frac{i}{\sqrt{8}}\zeta$}  & \multirow{3}{*}{$-\frac{1}{\sqrt{8}}\eta$} &\\\cline{1-5}
                11 & - & - & - & 0 & & &\\\cline{1-5}
                12 & - & - & - & + & & &\\\hline
                13 & 0 & - & - & + & \frac{1}{2}\eta & \frac{i}{2}\zeta \\\hline
                14 & + & - & - & + & \frac{(-1)^{m_1}}{\sqrt{8}}\eta & \frac{i(-1)^{m_1}}{\sqrt{8}}\zeta & \\\hline
                15 & + & - & 0 & + & -\frac{i(-1)^{m_1}}{2}\zeta & \frac{i (-1)^{m_1}}{2}\zeta & \\\hline
                16 & + & - & + & + & -\frac{i(-1)^{m_2}}{\sqrt{8}}\zeta & -\frac{(-1)^{m_2}}{\sqrt{8}}\eta & \\ \hline
            \end{tabular}
            \begin{tablenotes}
            \item $\eta=(-1)^{l_1+l_2-l}+1$,\quad $\zeta=(-1)^{l_1+l_2-l}-1$.
            \end{tablenotes}
        \end{threeparttable}

    \end{center}
\end{table}

From the orthogonality of $C$ and $T$, one can show
\begin{align}
    c_{l_1,l_2} \overline{c}_{l_1,l_2}^T = I.
\end{align}

From this formulation, we have
\begin{align}
    U^{l_1}(R) \otimes U^{l_2}(R) & = c_{l_1,l_2} \bracket{ \bigoplus_{l=|l_1-l_2|}^{l_1+l_2} U^l(R) } \overline{c}_{l_1,l_2}^T.
\end{align}
In the element-wise form,
\begin{align}
    U^{l_1}_{m_1,n_1} (R) U^{l_2}_{m_2,n_2}(R) = \sum_{l=|l_1-l_2|}^{l_1+l_2} \sum_{m,n=-l}^l c^{l,m}_{l_1,m_1,l_2,m_2} \overline{c}^{l,n}_{l_1,n_1,l_2,n_2} U^{l}_{m,n}(R).
\end{align}
From \eqref{eqn:c_lm_zero}, the summation reduces to
\begin{align}
    U^{l_1}_{m_1,n_1} (R) U^{l_2}_{m_2,n_2}(R) =  \sum_{m\in M} \sum_{n\in N} \sum_{l=\max\{|l_1-l_2|,|m|,|n|\}}^{l_1+l_2} c^{l,m}_{l_1,m_1,l_2,m_2} \overline{c}^{l,n}_{l_1,n_1,l_2,n_2} U^{l}_{m,n}(R).
\end{align}
where
\begin{align}
    M=\{m_1+m_2,m_1-m_2,-m_1+m_2,-m_1-m_2\},\quad N=\{n_1+n_2,n_1-n_2,-n_1+n_2,-n_1-n_2\}.
\end{align}
Interestingly, the Clebsch-Gordon coefficients for the real harmonics are complex in general. 
But, it appears that the term $ c^{l,m}_{l_1,m_1,l_2,m_2} \overline{c}^{l,n}_{l_1,n_1,l_2,n_2}$ is real always.

\bibliography{/Users/tylee/Documents/BibMaster17}
\bibliographystyle{IEEEtran}

\end{document}

% Motivated by these, the real part and the imaginary part of $D^l_{m,n}$ scaled by a factor $\sqrt{2}$ are defined as
% \begin{align*}
% \mathcal{R}^l_{m,n}=\frac{1}{\sqrt{2}} (D^l_{m,n}+\overline{D^l_{m,n}}) = \sqrt{2} d^l_{m,n}(\beta) \cos (m\alpha+n\gamma),\\
% \mathcal{I}^l_{m,n}=\frac{1}{\sqrt{2}i} (D^l_{m,n}-\overline{D^l_{m,n}}) = \sqrt{2} d^l_{m,n}(\beta) \sin (m\alpha+n\gamma).
% \end{align*}
% They are mutually orthogonal as
% \begin{align*}
% \pair{\mathcal{R}^{l_1}_{m_1,n_1},\mathcal{R}^{l_2}_{m_2,n_2}}
% &= \frac{1}{8\pi^2}\int_0^{2\pi}\int_{0}^\pi\int_0^{2\pi} 2 d^{l_1}_{m_1,n_1}(\beta)\cos(m_1\alpha+n_1\gamma)
% d^{l_2}_{m_2,n_2}(\beta)\cos(m_2\alpha+n_2\gamma)\sin\beta d\alpha d\beta d\gamma\\
% &=\frac{1}{2\pi^2} \frac{1}{2l_1+1} \delta_{l_1,l_2} \int_0^{2\pi}\int_0^{2\pi}\cos(m_1\alpha+n_1\gamma)
% \cos(m_2\alpha+n_2\gamma) d\alpha d\gamma\\
% &=\frac{1}{2l_1+1} \delta_{l_1,l_2}\delta_{m_1,m_2}\delta_{n_1,n_2}\\
% \pair{\mathcal{I}^{l_1}_{m_1,n_1},\mathcal{I}^{l_2}_{m_2,n_2}}
% &=\frac{1}{2l_1+1} \delta_{l_1,l_2}\delta_{m_1,m_2}\delta_{n_1,n_2}\\
% \pair{\mathcal{R}^{l_1}_{m_1,n_1},\mathcal{I}^{l_2}_{m_2,n_2}}
% &=0.
% \end{align*}
% 
% 
% 
% Similar with real spherical harmonics, we formulate the matrix representation for the real-valued functions as
% \begin{align}
% U^l_{m,n}(R) = \begin{cases}
% \mathcal{I}^l_{m,n}(I)=\sqrt{2} d^l_{m,n}(\beta) \sin (m\alpha+n\gamma) & (m,n)\in I^l_-,\\
% \mathcal{R}^l_{0,0}(R)=\sqrt{2} d^l_{0,0}(\beta) & (m,n)=(0,0),\\
% \mathcal{R}^l_{-m,-n}(R)=\sqrt{2} d^l_{-m,-n}(\beta) \cos (m\alpha+n\gamma) & (m,n)\in I^l_+.
% \end{cases}
% \end{align}
% Due to the orthogonality of $\mathcal{R}$ and $\mathcal{I}$, 
% \begin{equation}
% \pair{ U^{l_1}_{m_1,n_1}(R), U^{l_2}_{m_2,n_2}(R)} = \frac{1}{2l_1+1}\delta_{l_1,l_2}\delta_{m_1,m_2}\delta_{n_1,n_2}. \label{eqn:U_ortho}
% \end{equation}
% 
% And, a real-valued, band-limited function is expanded by
% \begin{align}
% f(R(\alpha,\beta,\gamma)) &= \sum_{l=0}^{B-1} \sum_{m,n=-l}^l (2l+1)\tilde f^l_{m,n} U^l_{m,n}(\alpha,\beta,\gamma).\label{eqn:fB_real}
% \end{align}
% The Fourier spectrum is calculated by
% \begin{align}
% \tilde f^l_{m,n} & = \pair{f(R(\alpha,\beta,\gamma)), U^l_{m,n}}\nonumber\\
% & =\frac{1}{8\pi^2}\int_0^{2\pi}\int_{0}^\pi\int_0^{2\pi} f(R(\alpha,\beta,\gamma)) U^{l}_{m,n}(R(\alpha,\beta,\gamma)) d\alpha d\beta d\gamma.\label{eqn:f_FT_real}
% \end{align}
% 
% However, $U$ is not a homomorphism, i.e., $U(R_1)U(R_2)\neq U(R_1R_2)$ in general. 
% 
% 
% \[
% R=
% \begin{bmatrix}
% \cos\beta \cos\alpha\cos\gamma - \sin\alpha\sin\gamma& - \cos\gamma\sin\alpha - \cos\alpha\cos\beta\sin\gamma& \cos\alpha\sin\beta\\
% \cos\alpha\sin\gamma + \cos\beta\cos\gamma\sin\alpha&   \cos\alpha\cos\gamma - \cos\beta\sin\alpha\sin\gamma& sin\alpha\sin\beta\\
% -\cos\gamma\sin\beta&	\sin\beta\sin\gamma &\cos\beta
% \end{bmatrix}
% \]
% 
% \[
% R=
% \begin{bmatrix}
% \cos\beta (c_{\alpha+\gamma}+c_{\alpha-\gamma}) +(c_{\alpha+\gamma}-c_{\alpha-\gamma})& - (s_{\alpha+\gamma}+s_{\alpha-\gamma}) - (s_{\alpha+\gamma}-s_{\alpha-\gamma})\cos\beta& \cos\alpha\sin\beta\\
% (s_{\alpha+\gamma}-s_{\alpha-\gamma}) + \cos\beta(s_{\alpha+\gamma}+s_{\alpha-\gamma})&   (c_{\alpha+\gamma}+c_{\alpha-\gamma}) + \cos\beta(c_{\alpha+\gamma}-c_{\alpha-\gamma})& \sin\alpha\sin\beta\\
% -\cos\gamma\sin\beta&	\sin\beta\sin\gamma &\cos\beta
% \end{bmatrix}
% \]
% 
% \begin{align*}
% 2\cos\alpha\cos\gamma  &=\cos(\alpha+\gamma)+\cos(\alpha-\gamma)=c_{\alpha+\gamma}+c_{\alpha-\gamma}\\
% 2\sin\alpha\sin\gamma  &=-\cos(\alpha+\gamma)+\cos(\alpha-\gamma)=-c_{\alpha+\gamma}+c_{\alpha-\gamma}\\
% 2\cos\alpha\sin\gamma  &=\sin(\alpha+\gamma)-\sin(\alpha-\gamma)=s_{\alpha+\gamma}-s_{\alpha-\gamma}\\
% 2\sin\alpha\cos\gamma  &=\sin(\alpha+\gamma)+\sin(\alpha-\gamma)=s_{\alpha+\gamma}+s_{\alpha-\gamma}\\
% \end{align*}
% 
% \begin{align*}
% R
% &=
% \begin{bmatrix}
% \frac{1}{2}(\mathcal{R}^1_{-1,-1}-\mathcal{R}^1_{1,-1}) & \frac{1}{2}(\mathcal{I}^1_{-1,-1}-\mathcal{I}^1_{1,-1}) & \sqrt{2}\mathcal{R}_{-1,0}\\
% \frac{1}{2}(-\mathcal{I}^1_{-1,-1}+\mathcal{I}^1_{-1,1})
% &\frac{1}{2}(\mathcal{R}^1_{-1,-1}+\mathcal{R}^1_{-1,1})
% &-\sqrt{2}\mathcal{I}_{-1,0}\\
% -\sqrt{2}\mathcal{R}_{0,-1}
% & \sqrt{2}\mathcal{I}_{0,-1}
% & \mathcal{R}_{0,0}
% \end{bmatrix}\\
% &=
% \begin{bmatrix}
% \frac{1}{2}(\mathcal{R}^1_{1,1}-\mathcal{R}^1_{-1,1}) 
% & \frac{1}{2}(-\mathcal{I}^1_{1,1}+\mathcal{I}^1_{-1,1}) 
% & -\sqrt{2}\mathcal{R}_{1,0}\\
% \frac{1}{2}(\mathcal{I}^1_{1,1}-\mathcal{I}^1_{1,-1})
% &  \frac{1}{2}(\mathcal{R}^1_{1,1}+\mathcal{R}^1_{1,-1})
% & -\sqrt{2}\mathcal{I}_{1,0}\\
% \sqrt{2}\mathcal{R}_{0,1}
% &\sqrt{2}\mathcal{I}_{0,1}
% & \mathcal{R}_{0,0}
% \end{bmatrix}\\
% &=
% \begin{bmatrix}
% \frac{1}{2}(\mathcal{R}^1_{-1,-1}-\mathcal{R}^1_{-1,1}) 
% & \frac{1}{2}(\mathcal{I}^1_{-1,-1}+\mathcal{I}^1_{-1,1}) 
% & \sqrt{2}\mathcal{R}_{-1,0}\\
% \frac{1}{2}(-\mathcal{I}^1_{-1,-1}+\mathcal{I}^1_{-1,1})
% &\frac{1}{2}(\mathcal{R}^1_{-1,-1}+\mathcal{R}^1_{-1,1})
% &-\sqrt{2}\mathcal{I}_{-1,0}\\
% -\sqrt{2}\mathcal{R}_{0,-1}
% & \sqrt{2}\mathcal{I}_{0,-1}
% & \mathcal{R}_{0,0}
% \end{bmatrix}\\
% \end{align*}
% 
% 
% \begin{align*}
% R&=
% \begin{bmatrix}
% \frac{1}{2}(\mathcal{R}^1_{-1,-1}-\mathcal{R}^1_{1,-1}) 
% & \sqrt{2}\mathcal{R}_{-1,0}
% & \frac{1}{2}(\mathcal{I}^1_{-1,-1}-\mathcal{I}^1_{1,-1}) \\
% -\sqrt{2}\mathcal{R}_{0,-1}
% & \mathcal{R}_{0,0}
% & \sqrt{2}\mathcal{I}_{0,-1}\\
% \frac{1}{2}(-\mathcal{I}^1_{-1,-1}+\mathcal{I}^1_{-1,1})
% &-\sqrt{2}\mathcal{I}_{-1,0}
% &\frac{1}{2}(\mathcal{R}^1_{-1,-1}+\mathcal{R}^1_{-1,1})
% \end{bmatrix}\\
% &=
% \begin{bmatrix}
% \frac{1}{2}(\mathcal{R}^1_{1,1}-\mathcal{R}^1_{-1,1}) 
% & -\sqrt{2}\mathcal{R}_{1,0}
% & \frac{1}{2}(-\mathcal{I}^1_{1,1}+\mathcal{I}^1_{-1,1}) \\
% \sqrt{2}\mathcal{R}_{0,1}
% & \mathcal{R}_{0,0}
% &\sqrt{2}\mathcal{I}_{0,1}\\
% \frac{1}{2}(\mathcal{I}^1_{1,1}-\mathcal{I}^1_{1,-1})
% & -\sqrt{2}\mathcal{I}_{-1,0}
% &  \frac{1}{2}(\mathcal{R}^1_{1,1}+\mathcal{R}^1_{1,-1})
% \end{bmatrix}
% \end{align*}
% 
% 
% \[
% d^1=\begin{bmatrix}
% \frac{1}{2}(1+\cos\beta) & \frac{1}{\sqrt{2}}\sin\beta & \frac{1}{2}(1-\cos\beta)\\
% -\frac{1}{\sqrt{2}}\sin\beta & \cos\beta & \frac{1}{\sqrt{2}}\sin\beta\\
% \frac{1}{2}(1-\cos\beta) & -\frac{1}{\sqrt{2}}\sin\beta & \frac{1}{2}(1+\cos\beta)
% \end{bmatrix}
% \]
% 
% \[
% D^1=\begin{bmatrix}
% \frac{1}{2}(1+\cos\beta)(\cos(\alpha+\gamma)-i\sin(\alpha+\gamma)) & \frac{1}{\sqrt{2}}\sin\beta(\cos\alpha-i\sin\alpha) & \frac{1}{2}(1-\cos\beta)(\cos(\alpha-\gamma)-i\sin(\alpha-\gamma))\\
% -\frac{1}{\sqrt{2}}\sin\beta(\cos\gamma-i\sin\gamma) & \cos\beta & \frac{1}{\sqrt{2}}\sin\beta(\cos\gamma+i\sin\gamma)\\
% \frac{1}{2}(1-\cos\beta)(\cos(\alpha-\gamma)+i\sin(\alpha-\gamma)) & -\frac{1}{\sqrt{2}}\sin\beta(\cos\alpha+i\sin\alpha) & \frac{1}{2}(1+\cos\beta)(\cos(\alpha+\gamma)+i\sin(\alpha+\gamma))
% \end{bmatrix}
% \]
% 
% \bibliography{/Users/tylee/Documents/BibMaster17}
% \bibliographystyle{IEEEtran}
% 
% \end{document}
